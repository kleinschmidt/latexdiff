%%%%%%%%%%%%%%%%%%%%%%%%%%%%%%%%%%%%%%%%%%%%%%%%%%%%%
%DIF LATEXDIFF DIFFERENCE FILE
%DIF DEL island_obs2004-old.tex   Sun Nov 11 21:19:29 2012
%DIF ADD island_obs2004-new.tex   Sun Nov 11 21:19:29 2012
% draft OBS experiment Iceland 2002, manuscript
% ueberareitet 21. sept. 2003
%DIF 4a4
%   latest changes FJT 27/2/2004 %DIF > 
%DIF -------
%%%%%%%%%%%%%%%%%%%%%%%%%%%%%%%%%%%%%%%%%%%%%%%%%%%%%
%\documentclass[extra,mreferee]{gji}
\documentclass{article}
\usepackage{graphicx}
%DIF 8c9
%DIF < %\usepackage{timet}
%DIF -------
\usepackage{pslatex} %DIF > 
%DIF -------

\providecommand{\chg}[1]{{\sf #1}}
\providecommand{\del}[1]{{[...]\footnote{\footnotesize removed: \sf #1}}}
%DIF 12c13-14
%DIF < \providecommand{\remark}[1]{{[\bf #1]}}
%DIF -------
\providecommand{\remark}[1]{{\typeout{REMARK (p.\arabic{page}): #1}[\bf #1]}} %DIF > 
%\renewcommand{\remark}[1]{{[\bf #1]}} %DIF > 
%DIF -------

\renewcommand{\includegraphics}[2][1]{{\sf Figure: {\tt #1}}}
\providecommand{\dg}{\ensuremath{^\circ}}

%---------------------------------------------------------
% \del und chg loeschen durch c-programm dechg
% siehe macro/dechg 
%
%\remark's werden nicht automatisch entfernt.  Du kannst
%im Code
%
%char gapstr[]="\\gap";
%
%durch
%
%char gapstr[]="\\remark";
%
%ersetzen, und neu kompilieren, dann geht das auch.
% Frederik
%-----------------------------------------------------------


\usepackage{a4,makeidx,natbib}
\citestyle{tectonophysics}

%\input{../../macro/format}       % macros, formate etc
%\input{../../macro/defin}
%DIF PREAMBLE EXTENSION ADDED BY LATEXDIFF
%DIF UNDERLINE PREAMBLE %DIF PREAMBLE
\RequirePackage[normalem]{ulem} %DIF PREAMBLE
\RequirePackage{color}\definecolor{RED}{rgb}{1,0,0}\definecolor{BLUE}{rgb}{0,0,1} %DIF PREAMBLE
\providecommand{\DIFadd}[1]{{\protect\color{blue}\uwave{#1}}} %DIF PREAMBLE
\providecommand{\DIFdel}[1]{{\protect\color{red}\sout{#1}}}                      %DIF PREAMBLE
%DIF SAFE PREAMBLE %DIF PREAMBLE
\providecommand{\DIFaddbegin}{} %DIF PREAMBLE
\providecommand{\DIFaddend}{} %DIF PREAMBLE
\providecommand{\DIFdelbegin}{} %DIF PREAMBLE
\providecommand{\DIFdelend}{} %DIF PREAMBLE
%DIF FLOATSAFE PREAMBLE %DIF PREAMBLE
\providecommand{\DIFaddFL}[1]{\DIFadd{#1}} %DIF PREAMBLE
\providecommand{\DIFdelFL}[1]{\DIFdel{#1}} %DIF PREAMBLE
\providecommand{\DIFaddbeginFL}{} %DIF PREAMBLE
\providecommand{\DIFaddendFL}{} %DIF PREAMBLE
\providecommand{\DIFdelbeginFL}{} %DIF PREAMBLE
\providecommand{\DIFdelendFL}{} %DIF PREAMBLE
%DIF END PREAMBLE EXTENSION ADDED BY LATEXDIFF

\begin{document}

%\maketitle
%---------------------
\begin{center}
{\Large\bf 
%DIF < Quantity, quality and predictability of 
\DIFdelbegin \DIFdel{Ocean Bottom Seismic Noise }\DIFdelend \DIFaddbegin \DIFadd{Seismic broadband ocean bottom data and noise observed with
free-fall stations:
experiences from long-term deployments }\DIFaddend in the North Atlantic and \DIFdelbegin \DIFdel{at other oceanic sites, 
and its relation to local and distant water gravity wave amplitudes
}%DIFDELCMD < \remark{
%DIFDELCMD < Seismic broadband ocean bottom data and noise observed with
%DIFDELCMD < free-fall stations:
%DIFDELCMD < experiences from long-term deployments in the North Atlantic and Tyrrhenian Sea
%DIFDELCMD < }
%DIFDELCMD < %%%
\DIFdelend \DIFaddbegin \DIFadd{Tyrrhenian Sea
}\DIFaddend }
\end{center}
%---------------------
{\bf T. Dahm, F. Tilmann, J.P. Morgan}

%---------------------
\section*{Abstract}
%---------------------
\DIFdelbegin %DIFDELCMD < \remark{noch zu ueberabeiten}
%DIFDELCMD < %%%
\DIFdel{Seismic noise on the ocean floor is mainly characterized by 
oceanic gravity-wave generated microseisms and cannot be avoided 
by moving a station some kilometres apart.
}\DIFdelend In a comparative study of two long-term deployments 
we \DIFdelbegin \DIFdel{are able to characterize the noise situation }\DIFdelend \DIFaddbegin \DIFadd{characterize the seismic 
noise }\DIFaddend on the seafloor in the North Atlantic  south of Iceland 
and in the Tyrrhenian Sea north of Sicily.
We estimate \DIFaddbegin \DIFadd{the teleseismic }\DIFaddend body-wave detection \DIFdelbegin \DIFdel{thresholds  at both sites and compare them 
to sites in the Pacific and to waveheights of oceanic gravity waves.
By means of a correlation technique between }\DIFdelend \DIFaddbegin \DIFadd{threshold to be
approximately at magnitude 5.5 at frequencies below the microseismic band ($f<0.1$
Hz) at the quietest sites in both regions. In the microseismic band
(0.1 Hz$<f<$<0.3 Hz), the minimum magnitudes for events to be recorded most of the
time are $M_W=6.0$ for the Tyrrhenian Sea deployment, and $M_W=7.5$
for the North Atlantic deployment.
By correlating }\DIFaddend seismic noise and 
oceanic \DIFdelbegin \DIFdel{waveheights }\DIFdelend \DIFaddbegin \DIFadd{waveheight amplitudes }\DIFaddend we are able to find the major generation 
areas of microseismic noise in the North Atlantic.
While the high noise of secondary microseisms at 0.24 $Hz$ is 
generated far away from the ocean bottom stations 
at three near-coastal \DIFdelbegin \DIFdel{sites}\DIFdelend \DIFaddbegin \DIFadd{regions}\DIFaddend , 
the microseismic noise at about 1 $Hz$ is generated directly at the stations.
We present a technique to estimate the noise generation areas prior 
to \DIFaddbegin \DIFadd{future }\DIFaddend deployment by using noise at nearby land-stations.
\\
The ambient low-frequency noise \DIFaddbegin \DIFadd{at }\DIFaddend 0.1 $Hz$ occurs mainly on horizontal 
components and is \DIFaddbegin \DIFadd{probably }\DIFaddend induced by seafloor-current induced tilt.
The  \DIFdelbegin \DIFdel{variability of the current-induced }\DIFdelend power spectral density \DIFdelbegin \DIFdel{noise
varies }\DIFdelend \DIFaddbegin \DIFadd{of this noise
varies by }\DIFaddend a factor of \DIFaddbegin \DIFadd{up to }\DIFaddend 10000 between different stations and deployment sites, 
indicating in some cases wobbling deployments, 
possible problems of frame weakness\DIFaddbegin \DIFadd{, }\DIFaddend and 
a possible higher noise-sensitivity of external packs to 
seafloor currents.
\DIFdelbegin \DIFdel{We identify that cross-coupling }\DIFdelend \DIFaddbegin \DIFadd{Cross-coupling }\DIFaddend between horizontal and vertical channel
noise is strong at some of our stations, 
demonstrating that the levelling mechanics can be further 
improved to reduce vertical channel noise.

%DIF < ---------------------
%DIF < \begin{keywords}  
%DIF <   keywords text
%DIF < \end{keywords}  
{{\bf keywords:}
ocean bottom seismology, seafloor seismic noise,
ocean water waves
%---------------------

\DIFaddbegin \remark{General point on units: there is no consistency whether to say 1~Hz or
1$Hz$.  I checked a random article in BSSA, and there it appears to be
1~Hz.  I have not corrected this anywhere, though.}

\remark{On a similar note: we might want to say OBS instead of obs. I
leave it to you to decide. In
the text I sometimes made the opposite correction for consistency reasons.}


\DIFaddend %---------------------
\section{Introduction}
%---------------------
%DIF <  Most of the known hot-spots and postulated mantle plumes are in oceanic 
%DIF <  regions.
%DIF <  The majority of active plate boundaries  are in regions covered by 
%DIF <  water. 
%DIF <  The systems of mid-ocean ridges can be considered as an elongated, 
%DIF <  about 70.000 km long volcano 
%DIF <  \chg{whose} eruption style and dynamics \chg{are still not known in detail}.
%DIF <  At submarine subduction zones a 
%DIF <  complex hierarchy of physical and chemical processes
%DIF <  is responsible for the continuous growth of the continental crust, 
%DIF <  as well as hazardous earthquakes and volcanic eruptions.  
%DIF <  This listing immediately shows that we cannot avoid to conduct geophysical 
%DIF <  experiments and measurements on the seafloor, when seriously 
%DIF <  aiming to gain a better
%DIF <  understanding of the geodynamic, tectonic, volcanic and chemical processes
%DIF <  on our Earth.
%DIF <  \remark{ABOVE PARAGRAPH READS MORE LIKE A PROPOSAL INTRO FOR RELATIVE LAY PEOPLE AND COULD BE SHORTENED TO ONE OR TWO SENTENCES FOR A PAPER IN A SPECIALIST JOURNAL (E.G. FOR BSSA)}
%DIF <  
The number of temporal deployments of broadband seismological arrays on the 
ocean bottom 
will rapidly increase in the near future, since 
the \DIFdelbegin \DIFdel{technique of such deployments }\DIFdelend \DIFaddbegin \DIFadd{technology  }\DIFaddend is already available and 
\DIFdelbegin \DIFdel{ocean bottom stations (obs) play a key-role 
for the understanding of different current questions, as:
How is oceanic crust formed and
subducted and how is 
material transport at these zones ?
How deep is the root of
mantle plumes or do they exist at all? 
For }\DIFdelend \DIFaddbegin \DIFadd{because the key processes of plate-formation at mid-ocean ridges and
plate-destruction at subduction zones occur wholly or partly
underneath the oceans. 
As another }\DIFaddend example, \cite{ritsema:03} 
point out that large aperture passive ocean bottom \DIFaddbegin \DIFadd{seismometer (obs)
}\DIFaddend experiments are necessary to give conclusive evidence 
for or against the existence of
whole mantle plumes\DIFaddbegin \DIFadd{, currently a point of strong controversy}\DIFaddend . 
Array apertures of at least 1000 km
are needed to resolve structures below 400 km depth.
Using a seismological array with an aperture of 300-500 km only, 
e.g. a land based array on Iceland, 
it is difficult to discriminate plumes from small-scale upper mantle
convection or normal thermal fluctuations.  
Islands associated with other oceanic hotspots are generally even smaller
than e.g. Iceland, resulting in even lower resolvable maximum depths there.
\DIFdelbegin %DIFDELCMD < \remark{evtl. andere aktuelle Initiativen f\"ur OBS Grossarrays erwaehnen, max 1 Satz}
%DIFDELCMD < %%%
\DIFdelend 

In spite of the obvious need for long-term seafloor 
deployments \DIFdelbegin \DIFdel{of seismic broadband arrays and 
the availability of the required technology for a number of years,
 }\DIFdelend so far these experiments 
have only rarely been carried out.
There are many reasons: 
first, a large technical and logistical effort is necessary 
to collect ocean bottom broadband data. 
Second, a large number of suitable 
ocean bottom stations is simply not available for the majority of 
seismologists.
A third point \DIFdelbegin \DIFdel{, however, }\DIFdelend is that the seismic noise on the seafloor is
often much larger than on land\DIFdelbegin \DIFdel{, and experiments are thus more difficult}\DIFdelend .
A magnitude 7 earthquake recorded at about $70^{\circ}$ epicentral 
distance may easily be hidden in the noise of a seafloor station,  
while it usually generates beautiful seismograms on a permanent 
land station at the same distance \DIFdelbegin \DIFdel{.
For example, \mbox{%DIFAUXCMD
\cite{hedlin:89}
}%DIFAUXCMD
made a comparative study of island, seafloor and sub-seafloor ambient noise
in the frequency range between 0.1 and 10 Hz.
They showed that the microseismic peak at the seafloor at about 
$0.25\, Hz$ appears to be 
as bad as on the poorest land stations which happen to be on small islands.
}\DIFdelend \DIFaddbegin \DIFadd{\mbox{%DIFAUXCMD
\cite{webb:98}
}%DIFAUXCMD
.
}\DIFaddend 

The general reasons for seafloor noise are well understood.
Since the early days of quantitative seismology a hundred years ago 
it has been suggested that oceanic gravity waves generate strong
noise signals,  
on the seafloor but also 
at inland stations hundred of kilometers from the coastlines. 
\DIFdelbegin \DIFdel{This type of seafloor noise typically shows two amplitude peaks 
at periods of about 0.1~Hz (}\DIFdelend \DIFaddbegin \remark{I have changed this sentence because we don't actually seem to
observe the two peaks}
\DIFadd{The }\DIFaddend primary microseismic peak \DIFdelbegin \DIFdel{) and 
in the range 0.2 - 0.4~Hz (secondary microseismic peak).
While the primary microseismic peak, }\DIFdelend \DIFaddbegin \DIFadd{at 0.1~Hz is not always visible, and,
}\DIFaddend although not well understood, 
is assumed to
be related to the interaction of oceanic waves and the coastlines\DIFdelbegin \DIFdel{, 
the secondary peak }\DIFdelend \DIFaddbegin \DIFadd{.
We only observe the secondary microseismic peak at 0.2--0.4 Hz, which }\DIFaddend is
thought to be generated by a nonlinear 
frequency doubling effect occurring  \DIFdelbegin \DIFdel{only when ocean }\DIFdelend \DIFaddbegin \DIFadd{when oceanic `swell' }\DIFaddend waves cross 
and standing waves develop 
\DIFdelbegin \DIFdel{\mbox{%DIFAUXCMD
\cite[e.g.][]{longuet-higgins:50,hasselmann:63,webb:98,essen:03}
}%DIFAUXCMD
.
This nonlinear effect is effective also in the deep sea, 
and it has been postulated that secondary microseisms may be generated 
during storms and possibly far from the coastlines}\DIFdelend \DIFaddbegin \DIFadd{\mbox{%DIFAUXCMD
\cite[e.g.,][]{longuet-higgins:50,hasselmann:63,webb:98,essen:03}
}%DIFAUXCMD
.
}\remark{Removed sentence because Webb, 1998, says that secondary
microseisms also preferentially generated in coastal areas because
only there you get crossing swell}
\DIFadd{Microseismic noise at higher frequencies is thought to be related to
crossing wind-driven waves \mbox{%DIFAUXCMD
\cite[e.g.,][]{webb:98}
}%DIFAUXCMD
}\DIFaddend .


It is of great interest to know the noise 
conditions of a region prior to planning an expensive ocean 
bottom experiment.   
For example, the experiment has to be planned for a sufficiently
long deployment period to obtain the required event
coverage. Alternatively, suitable targets or deployment periods with
low expected noise levels might be selected according to predicted
noise levels. In certain cases, a better knowledge of noise sources and
generation mechanisms might allow enhancement of the
signal-to-noise-ratio by processing, or guide the development of new
deployment strategies, e.g., by burying the sensor \DIFdelbegin \DIFdel{.
}%DIFDELCMD < \\
%DIFDELCMD < %%%
\DIFdel{For instance, 
\mbox{%DIFAUXCMD
\cite{collins:01}
}%DIFAUXCMD
summarize the findings from the 
Ocean Seismic Network Pilot Experiment, which was carried out 
over a period of about four months at a site 225 km southwest 
of Ohahu, Hawaii.
Their results demonstrate that the location of the seismometer, whether 
it be on the seafloor, superficially buried within the sea-bed, 
or in a deep borehole - has a profound effect on data quality.
At long-periods ($< 0.1\, Hz$), data quality was highest 
for a seismometer buried just below the seafloor, while at 
short-periods ($>0.1 \, Hz$), data quality was best for a seismometer
deployed 242 $m$ below the seafloor in a borehole.  
}%DIFDELCMD < \\
%DIFDELCMD < %%%
\DIFdel{\mbox{%DIFAUXCMD
\cite{crawford:00}
}%DIFAUXCMD
have deployed a long-period ($<0.1\, $Hz)
gravimeter and an STS2 sensor for a period of 14 days 
at the outer edge of the California continental border edge,
in 1~km distance to each other. 
Both station carried an additional differential pressure gauge.
They show that the low frequency noise ($<0.1\, $Hz) 
can be reduced by subtracting the part of the vertical seismogram 
coherent with the pressure records and the horizontal records. The
noise coherent with pressure is related to seafloor
compliance in response to waves between different layers in the water
column \mbox{%DIFAUXCMD
\cite[][]{webb:99}
}%DIFAUXCMD
, the noise coherent with the horizontal channels 
is related to instrument tilt \mbox{%DIFAUXCMD
\cite[e.g.][]{crawford:00}
}%DIFAUXCMD
.
}\DIFdelend \DIFaddbegin \DIFadd{\mbox{%DIFAUXCMD
\cite[e.g.,][]{collins:01}
}%DIFAUXCMD
.
}\DIFaddend 

In the following, we will 
give examples of \DIFdelbegin \DIFdel{wide-band OBS-recordings }\DIFdelend \DIFaddbegin \DIFadd{broad-band obs-recordings }\DIFaddend of teleseismic 
earthquakes measured at different sites 
during \DIFaddbegin \DIFadd{two }\DIFaddend different deployments.
Data are from free-fall \DIFdelbegin \DIFdel{OBS-ses }\DIFdelend \DIFaddbegin \DIFadd{obs }\DIFaddend from GEOMAR and Hamburg University.
A systematic comparison 
of  \DIFdelbegin \DIFdel{long-term }\DIFdelend noise-measurements reveals problems \DIFdelbegin \DIFdel{of the stations }\DIFdelend \DIFaddbegin \DIFadd{with the station }\DIFaddend design of some of our
\DIFdelbegin \DIFdel{OBS-ses}\DIFdelend \DIFaddbegin \DIFadd{obs}\DIFaddend , and helps to quantify detection thresholds for body- and surface waves.
Some of the deployed stations had strong noise 
\DIFdelbegin \DIFdel{on the vertical seismometer component 
}\DIFdelend at frequencies below 0.1 $Hz$, which could be attributed to
tilt-induced noise\DIFdelbegin \DIFdel{and 
could be corrected for.
We find that, in }\DIFdelend \DIFaddbegin \DIFadd{. This noise could partly be removed from the
vertical component using the horizontal components (see below).
In }\DIFaddend an extreme case, the sensor apparently has been 
tilted 
from the horizontal by about $7^{\circ}$ \DIFdelbegin \DIFdel{. 
}\DIFdelend \DIFaddbegin \DIFadd{or, less certain, even
$34^{\circ}$. }\remark{according to table 1}
\DIFaddend By using the correlation between seismic and ocean wave data we
confirm that microseismic noise at
frequencies at about 1~Hz is generated by oceanic wave action near the
station, whereas lower-frequency micro-seismic noise
($\sim$0.25~Hz) is correlated to wave action in few relatively narrow generation
areas, i.e. 
off-shore the North-West coast of
Scotland / Ireland, on the Reykjanes Ridge south of Iceland and in a band 
between North-Iceland and West-Norway.
The generation areas are dominant presumably because the
sub-marine topography and the preferential wind-wave directions 
are particularly favourable towards
conversion of oceanic wave energy into secondary microseismic energy.

%-----------------------------------------------
\section{Recordings of teleseismic earthquakes 
\DIFdelbegin \DIFdel{an }\DIFdelend \DIFaddbegin \DIFadd{and }\DIFaddend tilt-induced noise
}
%-----------------------------------------------
Between April and July 2002 
GEOMAR, Kiel, and the Institute of Geophysics, University of Hamburg, 
have installed a passive seismological network
in the North Atlantic south of Iceland
for three and a half months.
Ten free-fall stations were deployed, from which nine have been successfully recovered.
Four were 
Hamburg-type wide-band stations
(three component \DIFdelbegin \DIFdel{PMD }\DIFdelend \DIFaddbegin \DIFadd{PMD-113 }\DIFaddend seismic sensor and a piezoelectric 
hydrophone,
ob20, ob21, ob26, ob28),
the others GEOMAR-type stations, of which there were 
three four-component, (PMD seismometer and Differential Pressure
Gauge, DPG (ob24), PMD seismometer and 
 hydrophone (ob29), Spahr-Webb seismometer and
DPG (ob23)),
 and four 
one-component stations equipped  with a DPG (ob25\DIFdelbegin \DIFdel{.}\DIFdelend \DIFaddbegin \DIFadd{, }\DIFaddend ob29) or hydrophone
(ob22, ob27) only. 
Station design and configuration are described in more detail in 
\cite{dahm:02} and
\cite{flueh:99}.  \DIFdelbegin %DIFDELCMD < \\
%DIFDELCMD < %%%
\DIFdel{The }\DIFdelend \DIFaddbegin \DIFadd{The Spahr-Webb has a flat acceleration response and is thus less sensitive 
at low frequencies than the PMD sensor which has a flat velocity
response up to 0.02 Hz.
}

\DIFadd{The }\DIFaddend station depth varied between 540 meter at the youngest sample point 
beneath the Reykjanes Ridge (ob24) to  
2780 meter depth at the oldest point of the European plate (ob21).
The network was designed as a large aperture array of about 500 kilometres
and with a station-station 
distance of about 150 or more kilometres (Fig.~\ref{station_map}).
The goal was to test such a configuration as a pilot study for a
possible future
large tomographic study beneath the Iceland hotspot.
Body waves from earthquakes in the \DIFdelbegin \DIFdel{Aleutean}\DIFdelend \DIFaddbegin \DIFadd{Aleutian}\DIFaddend , Kamchatka  
or Japan region sample the depth range of about 500 $km$ beneath 
the Iceland hotspot before being recorded at the OBS station
\\


%DIF < Stations ob28, ob29 and ob20 were planned as a tripartite configuration 
%DIF < \cite[three stations at the edges of a equal-sided triangle, see][
%DIF < Fig.~\ref{station_map}]{gutenberg:47}. 
%DIF < A tripartite station allows an estimate of the back-azimuth of 
%DIF < plane waves stemming e.g. from regional or teleseismic generation 
%DIF < areas of microseisms, and may thus help to 
%DIF < separate earthquake signals 
%DIF < from microseismic noise.
%DIF < However, since ob20 could not be recovered for unknown reasons, 
%DIF < and ob29 functioned only for about 2 days, the mini-array could not be used 
%DIF < for signal enhancement.
%DIF < \\
\DIFdelbegin %DIFDELCMD < 

%DIFDELCMD < %%%
\DIFdelend Fig~\ref{kamchatka_M7.3}
shows the \DIFaddbegin \DIFadd{bandpass-filtered }\DIFaddend waveforms for a teleseismic M=7.3 earthquake in 
Kamchatka at a depth of $566\, km$ 
and epicentral distances of 72-74$^\circ$.
Seven stations were \DIFdelbegin \DIFdel{functional }\DIFdelend \DIFaddbegin \DIFadd{operational }\DIFaddend and recorded the body and surface waves\DIFdelbegin \DIFdel{with varying quality}\DIFdelend .
Station ob21 has \DIFdelbegin \DIFdel{, as in general, }\DIFdelend a fairly good signal-to-noise ratio
(SNR) on both, hydrophone and seismometer, 
and recorded a seismogram \DIFdelbegin \DIFdel{quite }\DIFdelend comparable to the seismogram from the 
broadband land station in Reykjavik (BORG).
 \DIFdelbegin \DIFdel{At }\DIFdelend \DIFaddbegin \DIFadd{The vertical channels of the  }\DIFaddend other Hamburg-type stations (ob26,
ob28) \DIFdelbegin \DIFdel{, the vertical channel }\DIFdelend \DIFaddbegin \DIFadd{are noisier }\DIFaddend below $0.1\, Hz$\DIFdelbegin \DIFdel{is more noisy compared to ob21}\DIFdelend .
A large portion of this low-frequency noise was tilt-induced and could be 
corrected for, as discussed below.
%DIF < All three Hamburg-type stations were of identical design and sensors. 
%DIF < However, station ob21 and ob26  
%DIF < had a relatively stiff frame made from 
%DIF < aluminum, while the frame of ob28 was more compliant and built from
%DIF < glass-fiber reinforced plastic (GRP or GFK).
%DIF < The noisy low-frequency record on the vertical channel of ob28 may partly
%DIF < stem from the compliant frame. 
The different noise observed on hydrophone channels at Hamburg-type 
OBS most likely reflects site effects.
\remark{kann man sagen, dass noise auf hydrophonen von Wassertiefe
abhaengt}
\DIFaddbegin \remark{Es sieht zwar so aus, allerdings hat man eigentlich nur drei
Datenpunkte.  Fuer die Infra-gravity waves wuerde man zwar solch einen
Trend erwarten, allerdings ist die Corner-Frequency fuer Infra-Gravity
noise zwisch 0.015 und 0.02 Hz, also weit unterhalb unserer
Beobachtung, ich wuerde mich da lieber nicht zuweit aus dem Fenster
lehnen}
\DIFaddend \\
Only ob23 of the GEOMAR stations recorded 4 channels for this event; 
the vertical channel has a comparable SNR  as Hamburg-type stations 
in this example.
The DPG records, which are sensitive at frequencies below 0.03 $Hz$, 
show large low-frequency noise apparently related to infragravity waves.
The GEOMAR hydrophones (ob22, ob27) \DIFdelbegin \DIFdel{do not recover 
earthquake signals for this event, for unknown reasons}\DIFdelend \DIFaddbegin \DIFadd{only recovered
earthquake signals at high frequencies above the corner frequency of
the bandpass filter used in the preparation of Figure~\ref{kamchatka_M7.3}}\DIFaddend .

During a six month period from December 2000 to May 2001 
the same \DIFdelbegin \DIFdel{type }\DIFdelend \DIFaddbegin \DIFadd{types }\DIFaddend of stations have been deployed in a pilot study 
in the Tyrrhenian Sea
\DIFdelbegin \DIFdel{\mbox{%DIFAUXCMD
\cite[e.g.][]{dahm:02}
}%DIFAUXCMD
}\DIFdelend \DIFaddbegin \DIFadd{\mbox{%DIFAUXCMD
\cite[e.g.,][]{dahm:02}
}%DIFAUXCMD
}\DIFaddend .
The OBS-array consisted of 14 stations with an aperture of about 
150 km and inter-station distances of about 25 kilometres.
The oceanic waveheight, and thus the microseismic noise peak, is much smaller
in the Tyrrhenian Sea than in the North Atlantic. 
As expected, the waveforms collected during this pilot study 
have a better SNR and the detection threshold for teleseismic body-waves
was much lower there.

Fig~\ref{quake_tysea}
shows a waveform example for a $M=6.7$ shallow earthquake at
$84^{\circ}$ epicentral distance.
\DIFdelbegin \DIFdel{All three seismometer components are plotted for a GEOMAR- (}\DIFdelend \DIFaddbegin \DIFadd{GEOMAR-type station }\DIFaddend ob05 \DIFdelbegin \DIFdel{) and 
two Hamburg-type stations.
Station ob05 was equipped similar to ob23 in the North Atlantic 
with a
different type of sensor
(}\DIFdelend \DIFaddbegin \DIFadd{was equipped with a
}\DIFaddend Spahr-Webb \DIFdelbegin \DIFdel{), which 
has a flat acceleration response and is thus less sensitive 
at frequencies below
0.07~$Hz$ than the PMD113 sensor of the }\DIFdelend \DIFaddbegin \DIFadd{sensor, and }\DIFaddend Hamburg-type stations \DIFdelbegin \DIFdel{. 
However, at least at }\DIFdelend ob10 and ob11 \DIFaddbegin \DIFadd{were
equipped with PMD sensors.
At least on ob10 and ob11}\DIFaddend , even horizontal channels 
recorded acceptable seismograms\DIFdelbegin \DIFdel{for this event, and }\DIFdelend \DIFaddbegin \DIFadd{, which }\DIFaddend are suitable for analysis of SKS shear-wave splitting 
(R\"umpker, pers. communication)
and receiver functions \cite[and pers. communication]{thorwart:04}.
\DIFaddbegin \remark{reference thorwart:04 does not exist}
\DIFaddend 

%---------------------------
% new: tilt removal section
%---------------------------
It is well known that a bad levelling of seismic sensors generates
cross-coupling between horizontal and vertical channels.
On the seafloor\DIFaddbegin \DIFadd{, }\DIFaddend the noise on the horizontal channels below 0.1 $Hz$ is 
often a factor of ten or more higher than on the vertical.
It is probably induced 
by seafloor current-induced 
\DIFdelbegin \DIFdel{displacement and }\DIFdelend tilt.
For a \DIFdelbegin \DIFdel{bad-levelled }\DIFdelend \DIFaddbegin \DIFadd{poorly levelled }\DIFaddend sensor the horizontal noise is transferred to the z-component\DIFdelbegin \DIFdel{and potentially masking earthquake signals there}\DIFdelend .
Then, the correlation and the coherency between noise measured on 
the vertical and horizontal channels is high.
\cite{crawford:00} and \cite{stutzmann:01} have demonstrated that the tilt-noise on the 
vertical channel below \DIFdelbegin \DIFdel{$0.1\, Hz$ }\DIFdelend can 
be quite efficiently reduced \DIFdelbegin \DIFdel{when 
}\DIFdelend \DIFaddbegin \DIFadd{by 
}\DIFaddend subtracting the cross-over signal predicted from 
the horizontal recordings. 
Their technique  \DIFdelbegin \DIFdel{is expected to work properly }\DIFdelend \DIFaddbegin \DIFadd{works well  }\DIFaddend when 
horizontal components are recording 
mainly current-induced noise (tilt noise).
As a by-product it \DIFdelbegin \DIFdel{reveals }\DIFdelend \DIFaddbegin \DIFadd{provides }\DIFaddend an estimate of the actual tilt angle of the 
sensor on the seafloor.
In Appendix A we briefly describe \DIFdelbegin \DIFdel{the }\DIFdelend \DIFaddbegin \DIFadd{their }\DIFaddend technique as implemented \DIFdelbegin \DIFdel{in our cases.
When applying the method of \mbox{%DIFAUXCMD
\cite{crawford:00}
}%DIFAUXCMD
and 
\mbox{%DIFAUXCMD
\cite{webb:99}
}%DIFAUXCMD
we 
assume that all three seismometer channels 
have an identical instrument sensitivity and worked 
properly. 
Another assumption is that the relative large noise on the 
horizontal channels is dominated by current-induced tilt.
At low frequencies it should be uncorrelated  to 
a precisely levelled vertical channel.
}\DIFdelend \DIFaddbegin \DIFadd{here.
}\DIFaddend 

\DIFaddbegin \remark{I have removed the list of assumptions because a. We don't
need to assume that horizontal noise is dominated by tilt, the
analysis will tell us whether it is b. We know the gain of the Hamburg stations.  Difference between
horizontal and vertical gain seems to be of the order of 10\% or so,
resulting in a similar error in our tilt estimate.  This is less than
the error we quote so we can ignore the issue of gain. }


\DIFaddend Inspection of the recordings 
at ob28 for a magnitude 6.5 earthquake show the
strong improvement affected by the tilt correction
(Fig.~\ref{fig:tilt-correction2}a,b).  No arrivals are visible on the
uncorrected record, but $P$, $PP$, $S$, $SS$, further phases, and
surface waves are clearly visible on the corrected trace.
At ob28, 
\DIFdelbegin \DIFdel{Power }\DIFdelend \DIFaddbegin \DIFadd{power }\DIFaddend spectral density (PSD) SNRs 
averaged over a number of earthquakes are improved by factors of
10-100 for the vertical channel but they still fall slightly short of
the signal quality achieved by station ob21  
(Fig.~\ref{fig:tilt-correction2}\DIFdelbegin \DIFdel{d). }\DIFdelend \DIFaddbegin \DIFadd{c). \mbox{%DIFAUXCMD
\citet{webb:99}
}%DIFAUXCMD
have introduced an
equivalent technique, henceforth called pressure correction, for removing correlated noise on the pressure and
vertical component.
}\DIFaddend However, the pressure
correction did not even result in a marginal improvement of the SNR,
because the pressure-vertical transfer function is of the same order
of magnitude for both noise and seismic arrivals such
that the noise correction also tends to reduce signal amplitudes.
Pressure-vertical noise coherence is also expected to be high at
frequencies well below the micro-seismic band because of seafloor 
compliance induced by infra-gravity
waves \citep{webb:99}. 
At station ob28 at a water depth of 2268 m, the
infra-gravity noise corner frequency
is expected at 0.015~Hz . 
%DIF < \remark{CALCULATE AFTER WEBB AND
%DIF <   CRAWFORD, 1999: $t_{\mbox{corner}}=\frac{3}{2}\sqrt{g/h} / 2 \pi$
%DIF < WITH $h$ THE WATER DEPTH AND $\frac{3}{2}$ BEING A FUDGE FACTOR TO GET
%DIF < THE SAME CORNER AS WEBB\&CRAWFORD} 
The pressure sensor of this instrument
only has a very small sensitivity at this frequency.  The same is true
for 
the other stations equipped with a PMD sensor 
(ob10, ob11, ob21, ob26 and ob29).  
Ob23, ob05 and ob06 were equipped with a DPG 
with larger sensitivity at $0.015\, Hz$ but their
seismometer sensitivity is low at low frequencies.  

\DIFaddbegin \DIFadd{For the following we applied the correction to stations  only 
for which an improvement in SNR was achieved 
(ob06, ob08, ob23, ob26, ob28, ob29), 
and discarded the pressure correction because it was 
never successful.
At the other stations (ob05, ob10, ob11, ob21)
we continued to use the uncorrected signals.  
}


\DIFaddend %------------------------- table 1 --------------------------------
\begin{table}
\begin{center}
\caption{
a) The Tyrrhenian Sea deployment 2000/2001. 
The apparent tilt-angle $\Theta$ between ${\bf g}$ and the 
vertical sensitivity axis of the sensor 
and the azimuth angle $\varphi$ between the tilt and the y-axis of the 
sensor \DIFaddbeginFL \DIFaddFL{($0<\varphi<90\dg$)}\DIFaddendFL . \DIFaddbeginFL \remark{or did you look at the sign as
well, and they just all happened to be less than 90\dg}\DIFaddFL{.
}\DIFaddendFL Both angles have been averaged from the ''flat'' 
part of the transfer functions $X/Z$, $Y/Z$ between
$0.025 \, Hz \, < \, f \, < \, 0.08 \, Hz$.
}
\vskip0.5cm
\label{tilt_table}
\begin{tabular}{|l||l|l|l|l|l|}
\hline
% &\multicolumn{6}{|c|}{Hamburg-type, integrated} & \multicolumn{4}{|c|}{Geomar-type, external pack}\\
% & ob08 & ob10 & ob11 & ob21 & ob26 & ob28 & ob05 & ob06 & ob23 & ob29 \\
& ob05 & ob06 & ob08 & ob10 & ob11 \\
\hline
$\Theta$ (deg) &  
 $0.3  \pm 0.6$ &
 $34  \pm 7.1  $ &
 $2.1  \pm 0.2 $ &
 $1.2  \pm 0.7 $ &
 $1.3  \pm 0.3 $ 
\\
\hline
$\varphi$ (deg) &  
 $7 \pm  4$ &
 $62 \pm  4 $ &
 $15  \pm  1 $ &
 $47 \pm 13 $ &
 $31 \pm  8 $ 
\\
\hline
\end{tabular}
\\
\vskip0.5cm
b) the same for the North Atlantic deployment 2002
\DIFaddbeginFL 

\remark{ob29: as the exchange of components is rather a trivial
mistake I would rather report 4\dg as the tilt, and then one should
also get more reasonable values for the tilt error and $\varphi$.}

\DIFaddendFL \vskip0.5cm
\begin{tabular}{|l||l|l|l|l|l|}
\hline
% &\multicolumn{6}{|c|}{Hamburg-type, integrated} & \multicolumn{4}{|c|}{Geomar-type, external pack}\\
% & ob08 & ob10 & ob11 & ob21 & ob26 & ob28 & ob05 & ob06 & ob23 & ob29 \\
& ob21 & ob23 & ob26 & ob28 & ob29 \\
\hline
$\Theta$ (deg) &  
 $0.3 \pm 0.2  $ &
 $0.7 \pm 0.6 $ &
 $0.8 \pm 0.2 $ &
 $7.1  \pm  0.7 $ &
 $86 \pm  72  $ 
\\
\hline
$\varphi$ (deg) &  
 $36 \pm  8 $ &
 $51 \pm  6 $ &
 $67\pm 12$ &
 $64 \pm  2 $ &
 $0 \pm 0 $ 
\\
\hline
\end{tabular}
\end{center}
\end{table}
% name      tilt     error     av-strike    error 
% ob05 0.29029  0.0646617 6.68699  4.30555   KOHAERENZ ZU KLEIN
% ob06 34.2118  7.11796   62.1155  3.81081
% ob08 2.14166  0.156842  14.962   0.827603
% ob10 1.22455  0.736119  47.3335 12.8416
% ob11 1.28882  0.278866  30.8946  7.71436
% ob21 0.300755 0.22498   35.7357  8.46431   Kohaerenz zu klein
% ob23 0.692174 0.567905  50.5514  5.92123
% ob26 0.767391 0.181256  66.9169 11.5758
% ob28 7.1288   0.682464  63.6499  1.84883
% ob29 86.4385  72.2217   0.143846 0.010359
%--------------------------------------------------------------------
Table~\ref{tilt_table} summarizes the estimated sensor 
tilt on the seafloor for the stations deployed in the 
Tyrrhenian Sea and the North Atlantic.
An apparent  station tilt of $\Theta < 0.3^{\circ}$
was found to be small enough to avoid significant cross-over noise.
Coherency between vertical and horizontal channel was then 
typically \DIFdelbegin \DIFdel{much }\DIFdelend \DIFaddbegin \DIFadd{well }\DIFaddend below 0.5. \DIFdelbegin \DIFdel{It was estimated }\DIFdelend \DIFaddbegin \DIFadd{This was the case
}\DIFaddend for only two of the 
ten \DIFdelbegin \DIFdel{functional four-channel station of two deployments.
This already }\DIFdelend \DIFaddbegin \DIFadd{fully operational four-component stations of the two deployments, which
 }\DIFaddend indicates that the levelling mechanism or the 
deployment technique 
is not satisfactory for \DIFdelbegin \DIFdel{both types of stations}\DIFdelend \DIFaddbegin \DIFadd{either type of station}\DIFaddend .
Hamburg-type stations have passive gimbal system where the 
seismic sensor is put in a highly viscous oil 
($10^5$\DIFaddbegin \DIFadd{Pa$\,$s (at $20^{\circ}\, C$) }\DIFaddend for the Tyrrhenian Sea and $10^6 \, Pa\,s$ \DIFdelbegin \DIFdel{at $20^{\circ}\, C$ for North Atlantic}\DIFdelend \DIFaddbegin \DIFadd{for the North Atlantic)}\DIFaddend .
Assuming that the internal sensitivity axis of the PMD sensors are 
aligned \DIFdelbegin \DIFdel{in }\DIFdelend \DIFaddbegin \DIFadd{at }\DIFaddend high precision, 
the observed mis-levelling of up to $7^{\circ}$ at ob28 
may have resulted from 
an \DIFdelbegin \DIFdel{non-precise }\DIFdelend \DIFaddbegin \DIFadd{imprecise }\DIFaddend weight balancing of the sensors in
the pendulum mechanics. 
GEOMAR-type sensors (Spahr Webb at ob05, ob06, ob23) 
have been actively gimbaled by a system developed \DIFdelbegin \DIFdel{from }\DIFdelend \DIFaddbegin \DIFadd{by }\DIFaddend Scripps.
The small tilt angle of $0.3^{\circ}$
and $0.7^{\circ}$ at ob05 and ob23, respectively, indicates
the in-situ precision of the mechanics when the external pack 
is properly placed on the seafloor.
In contrast, the \DIFdelbegin \DIFdel{extreme }\DIFdelend \DIFaddbegin \DIFadd{extremely }\DIFaddend large value
of e.g. $34^{\circ}$ at \DIFdelbegin \DIFdel{ob26 }\DIFdelend \DIFaddbegin \DIFadd{ob06 }\DIFaddend might be due to a strongly 
tilted external pack; a problem that has been observed also 
at other deployments with GEOMAR-type stations using an 
autonomously deploying external pack. \DIFdelbegin \DIFdel{The large angle of $86^{\circ}$ at ob29
(PMD sensor, passively gimbaled in viscous oil) 
showed us that the sensor connectors had been flipped and 
horizontal (x, channel 3) and vertical (z, channel 2 ) 
had been interchanged.
The correct tilt angle was about 4$^{\circ}$.
In the following we re-attributed the correct channels
to ob29.
The large tilt }\DIFdelend \DIFaddbegin \remark{The last statement could
do with a reference, even if it is pers. comm.  Also I think, although I am
not entirely sure, that the Webb gimbaling system works through a very
large angle (more than 34\dg)-it might be worth checking this with
Ernst or Joerg.  If so, the large angle must be due to a failure of
the self-levelling or very funny gains}
\remark{Removed paragraph does not need to be in paper (might be in
report)- rather just report the correct angle (4deg)}

\DIFadd{The large tilt }\DIFaddend angles and the large tilt-induced vertical channel 
noise found at several of our free-fall stations \DIFdelbegin \DIFdel{shows  
that our levelling mechanics has to be improved }\DIFdelend \DIFaddbegin \DIFadd{show
that the levelling mechanics }\DIFaddend as well 
as the deployment technique \DIFdelbegin \DIFdel{. 
A video-controlled }\DIFdelend \DIFaddbegin \DIFadd{clearly have room for improvement. 
A more controlled }\DIFaddend launching of stations \DIFaddbegin \DIFadd{aided by video images of the seafloor }\DIFaddend might be 
preferential to the \DIFdelbegin \DIFdel{so-far }\DIFdelend \DIFaddbegin \DIFadd{commonly }\DIFaddend used 
free-falling technique \DIFdelbegin \DIFdel{.
}\DIFdelend \DIFaddbegin \DIFadd{in order to ensure a stable, approximately
level placement of the stations on the seafloor.}\remark{I am still not
entirely sure what you mean: Do you envision lowering the instrument
to the seafloor on a cable and then slowly dragging it until the video
image shows a good location?}
\DIFaddend However, the problem of tilted sensors explains the 
cross-over noise from the horizontal to the vertical 
channel. It does not explain the high noise level on
horizontal channels. 
In the next section we compare noise levels
on vertical and horizontal channels.






\DIFdelbegin \DIFdel{For the following we applied the correction to stations  only 
for which an improvement in SNR was achieved 
(ob06, ob08, ob23, ob26, ob28, ob29), 
and discarded the pressure correction because it was 
never successful.
At the other stations (ob05, ob10, ob11, ob21)
we continued to use the uncorrected signals.  
}%DIFDELCMD < 

%DIFDELCMD < %%%
\DIFdelend %---------------------
\section{Comparison of noise measurements}
%---------------------
%--------------------------------------------
% average noise measurements 
%--------------------------------------------
A standard procedure to quantify the fidelity of a seismic station
is to compare average power spectral densities (PSD) 
with those
from good or poor permanent land stations all over the world, 
as for \DIFdelbegin \DIFdel{instance }\DIFdelend \DIFaddbegin \DIFadd{example }\DIFaddend given in the low- and high-noise model 
of \cite{peterson:93}.

The average PSD of background noise at our stations 
has been calculated by \DIFdelbegin \DIFdel{removing 
a three hour time window from the continuous recordings after every global $M \geq 5.5$ earthquake before averaging. One hour time intervals have been averaged
for the North Atlantic (April to July 2002) and the Tyrrhenian Sea deployment 
(December 2000 to May 2001)
and compared with }\DIFdelend \DIFaddbegin \DIFadd{averaging the PSD of one hour time intervals,
but excluding three hours after every earthquake with $M \geq
5.5$. Figure ~\ref{psd_comparison} compares the average PSD of the two deployments 
to }\DIFaddend the noise models of \cite{peterson:93}\DIFdelbegin \DIFdel{(Fig.
~\ref{psd_comparison}).
}\DIFdelend \DIFaddbegin \DIFadd{.
}\DIFaddend Below 0.1 $Hz$
the tilt-induced noise has been reduced for
ob08, ob23, ob26, ob28 and ob29 by the technique 
described above.
The PSD are characterized by three noise 
intervals;
the so-called ''low noise notch'' between 0.01 and 0.1 $Hz$ 
\DIFdelbegin \DIFdel{\mbox{%DIFAUXCMD
\cite[e.g.][]{webb:98}
}%DIFAUXCMD
}\DIFdelend \DIFaddbegin \DIFadd{\mbox{%DIFAUXCMD
\cite[e.g.,][]{webb:98}
}%DIFAUXCMD
}\DIFaddend , 
the microseismic noise between 0.1 and about 1 $Hz$ and 
the \DIFaddbegin \DIFadd{high frequency  }\DIFaddend noise above 1 $Hz$.
The stations with the lowest noise-levels in the frequency range between 
0.01 and 0.1 $Hz$ have $10^{-15} \, m^2 / s^4 / Hz$ and are close to
quiet stations of the low noise model of \cite{peterson:93}.

%possibly related to the passive 
%gimbaling system used, although a contribution of an 
%increased electronic noise of the sensor is expected too.
We believe that the $10^{-15} \, m^2 / s^4 / Hz$ noise level at 
0.05 $Hz$ is nearly the optimum for the used free-fall 
stations and is apparently independent of the region. 

The microseismic noise peak at 0.24 $Hz$ 
has a maximum of  
about $10^{-9} \, m^2 / s^4 / Hz$ in the North Atlantic and 
about $10^{-11} \, m^2 / s^4 / Hz$ in the Tyrrhenian Sea. 
It is in both cases the most dominant signal below 5 $Hz$.
The differences between the two deployments reflects 
differences in strength of the noise source and, 
may be to a larger part, differences in 
source-station distances. 
The microseismic noise measured in the North Atlantic is 
partly above the level of noisy oceanic island stations of
Petersons 
high noise model.

%As discussed below the deep seafloor noise at about $1\, Hz$ is correlated
%to oceanic wave-height directly above the station 
%and thus probably ocean-wave induced.
Noise and noise peaks above 2 $Hz$ are interpreted as 
local site resonances and shear-wave resonances as proposed by 
e.g.  \cite{godin:99}.
Since local and regional earthquakes have not been systematically 
deleted from the waveforms their signals 
may have added to the noise level observed in 
Fig.~\ref{psd_comparison}.

%--------------------------------------------
% psd horizontal traces
%--------------------------------------------
%------------------------- table 2 --------------------------------
\begin{table}
\begin{center}
\caption{
Average psd noise between $0.05 < f < 0.1\, Hz$ 
on horizontal channels x and y in 
$10^{-12} \, m^2/s^2 / Hz$ for the 
Tyrrhenian Sea deployment 2000/2001.
}
\vskip0.5cm
\label{hnoise_table}
\begin{tabular}{|l||l|l|l|l|l|}
\hline
& ob05 & ob06 & ob08 & ob10 & ob11 \\
\hline
x  &  
 $2162$ &
 $1148$ &
 $108 $ &
 $2  $ &
 $0.2   $ 
\\
\hline
y &  
 $2412$ &
 $1300$ &
 $1414$ &
 $2   $ &
 $0.3    $ 
\\
\hline
\end{tabular}
\\
\vskip0.5cm
b) the same for the North Atlantic deployment \DIFdelbeginFL \DIFdelFL{2002
}\DIFdelendFL \DIFaddbeginFL \DIFaddFL{2002. The large noise on the y-channel of ob29 is most likely not real but 
related to an enhanced sensitivity of the y-channel, 
}\DIFaddendFL \vskip0.5cm
\begin{tabular}{|l||l|l|l|l|l|}
\hline
& ob21 & ob23 & ob26 & ob28 & ob29 \\
\hline
x &  
 $2  $ &
 $5\cdot 10^5 $ &
 $194  $ &
 $1668 $ &
 $116  $ 
\\
\hline
y &  
 $3  $ &
 $3\cdot 10^5$ &
 $111 $ &
 $335 $ &
 $6\cdot 10^4$ 
\\
\hline
\end{tabular}
\end{center}
\DIFaddbeginFL \remark{Value for ob23 and ob29 seems not likely.  Why is x component
so different form y component for ob29}
\DIFaddendFL \end{table}
%%%%%%%%%%%%%%%%%%%%%%%%%%%%%%%%%%%%%%%%%%%%%%%%%%%%%%%
%psd noise innerhalb f1=0.05 (20 sec) und f2 = 0.1 Hz
%
%Spahr Webb wurde um nm Faktor (10E-18) und zusaetzlich emprisch
%um Faktor 5E-4 verringert (5. Spalte falls zutreffend). 
%Emprische Faktir durch visuelles
%Angleichen des Mikroseismischen Peaks bei ob06 psd Plot
%
%ob05/psd_2_337_117.dat 0.05 0.10 3.38613e+11 1.69306e-10
%ob05/psd_3_337_117.dat 0.05 0.10 4.32479e+12 2.1624e-09
%ob05/psd_4_337_117.dat 0.05 0.10 4.82532e+12 2.41266e-09
%
%ob06/psd_2_337_117.dat 0.05 0.10 9.96149e+09 4.98075e-12
%ob06/psd_3_337_117.dat 0.05 0.10 2.29625e+12 1.14812e-09
%ob06/psd_4_337_117.dat 0.05 0.10 2.60053e+12 1.30026e-09
%
%ob08/psd_2_337_117.dat 0.05 0.10 7.98247e-14
%ob08/psd_3_337_117.dat 0.05 0.10 1.08168e-10
%ob08/psd_4_337_117.dat 0.05 0.10 1.41438e-09
%
%ob10/psd_2_337_117.dat 0.05 0.10 2.38762e-15
%ob10/psd_3_337_117.dat 0.05 0.10 1.51226e-12
%ob10/psd_4_337_117.dat 0.05 0.10 2.24375e-12
%
%ob11/psd_2_337_117.dat 0.05 0.10 1.19714e-15
%ob11/psd_3_337_117.dat 0.05 0.10 1.21081e-13
%ob11/psd_4_337_117.dat 0.05 0.10 2.91109e-13
%
%ob21/psd_2_104_189.dat 0.05 0.10 9.80502e-16
%ob21/psd_3_104_189.dat 0.05 0.10 1.84611e-12
%ob21/psd_4_104_189.dat 0.05 0.10 2.61009e-12
%
%ob23/psd_2_104_189.dat 0.05 0.10 2.21491e+12 1.10745e-09
%ob23/psd_3_104_189.dat 0.05 0.10 1.00583e+15 5.02916e-07
%ob23/psd_4_104_189.dat 0.05 0.10 6.36158e+14 3.18079e-07

%ob26/psd_2_104_189.dat 0.05 0.10 2.48591e-14
%ob26/psd_3_104_189.dat 0.05 0.10 1.94155e-10
%ob26/psd_4_104_189.dat 0.05 0.10 1.11513e-10
%
%ob28/psd_2_104_189.dat 0.05 0.10 1.11646e-11
%ob28/psd_3_104_189.dat 0.05 0.10 1.66882e-09
%ob28/psd_4_104_189.dat 0.05 0.10 3.35145e-10
%
%ob29/psd_2_103_117.dat 0.05 0.10 1.16639e-10  vertauscht: z = x
%ob29/psd_3_103_117.dat 0.05 0.10 8.62627e-14  vertauscht: x = z
%ob29/psd_4_103_117.dat 0.05 0.10 6.23365e-08
%%%%%%%%%%%%%%%%%%%%%%%%%%%%%%%%%%%%%%%%%%%%%%%%%%%%%%%%%%%%%%%5
%--------------------------------------------------------------------
Table~\ref{hnoise_table} gives 
measured average noise-levels on horizontal channels in the 
frequency band from $0.03 - 0.1\, Hz$. \DIFdelbegin \DIFdel{Examples of }\DIFdelend \DIFaddbegin \remark{Table caption lower frequency bound of 0.05 Hz is in
contradiction to text where it says 0.03 Hz}
\DIFadd{Examples of frequency-dependent }\DIFaddend psd on horizontal channels are given in 
Fig.~\ref{fig:tilt-correction1}\DIFdelbegin \DIFdel{in Appendix A}\DIFdelend . 
The lowest level is at about $10^{-13}\, m^2/s^2 / Hz$
and thus a factor of 100 larger than the lowest level on 
vertical channels (factor of 10 in amplitudes). 
%DIF < -------------
\DIFaddbegin 

\DIFaddend The higher low-frequency noise level on horizontal channels is  
commonly observed on the seafloor and associated with 
current-induced tilt-transients experienced by the sensor. 
\DIFdelbegin \DIFdel{\mbox{%DIFAUXCMD
\cite{duennebier:95,crawford:00}
}%DIFAUXCMD
gives among others 
formulas to estimate  the tilt-effect on }\DIFdelend \DIFaddbegin \DIFadd{\mbox{%DIFAUXCMD
\cite{duennebier:95}
}%DIFAUXCMD
and \mbox{%DIFAUXCMD
\cite{crawford:00}
}%DIFAUXCMD
give the equations
to estimate  }\DIFaddend vertical and horizontal \DIFdelbegin \DIFdel{channels
as a function of an }\DIFdelend \DIFaddbegin \DIFadd{channel noise resulting from 
a }\DIFaddend harmonic tilt excitation with amplitude $\epsilon$,
frequency $\omega$\DIFdelbegin \DIFdel{and of station parameters 
as the 
}\DIFdelend \DIFaddbegin \DIFadd{,  and the station parameters 
}\DIFaddend distance $L$ of the sensor to the center of rotation\DIFdelbegin \DIFdel{.
The }\DIFdelend \DIFaddbegin \DIFadd{, and various angles.
At low frequencies, the }\DIFaddend tilt amplitude on the horizontal channels is \DIFdelbegin \DIFdel{at low frequencies }\DIFdelend to
first order proportional to $\epsilon$. 
If the center of rotation is far from the sensor, an additional 
term proportional to \DIFdelbegin \DIFdel{$L$ will add}\DIFdelend \DIFaddbegin \DIFadd{$\epsilon \omega^2 L$ becomes important at high frequencies}\DIFaddend .
A station or an instrument pack with a larger flow resistance torque
will \DIFdelbegin \DIFdel{thus generate larger current-induce }\DIFdelend \DIFaddbegin \DIFadd{generate larger current-induced }\DIFaddend tilt.
Further, the anchor of the station (instrument pack) 
will be important, since 
it has to couple on ground without wobbling 
and needs ground contact points spread \DIFdelbegin \DIFdel{a larger distance }\DIFdelend \DIFaddbegin \DIFadd{over as large distance as possible }\DIFaddend compared to 
station or package height. 
%-------------
\DIFdelbegin \DIFdel{An interesting questions }\DIFdelend \DIFaddbegin 

\DIFadd{An interesting question }\DIFaddend is how variable the noise 
on horizontal channels is and whether \DIFaddbegin \DIFadd{a }\DIFaddend systematic 
effect can be seen depending in the station type,
i.e. external pack (GEOMAR-type) compared to 
station-integrated sensor (Hamburg-type). 
In general, stations with the smallest noise on horizontal channels 
were the stations characterised with the highest fidelity 
for the vertical channel (ob10, ob11, ob21).
The problematic stations (ob08, ob28) were the stations with
\DIFdelbegin \DIFdel{an increased 
}\DIFdelend horizontal noise level \DIFdelbegin \DIFdel{of }\DIFdelend more than a factor of 1000 \DIFaddbegin \DIFadd{higher }\DIFaddend compared to
high-fidelity stations. 
Ob28 had a compliant frame built from GRP, that might have added to
an enhanced tilt-signal at this stations.
Otherwise, the increased tilt-noise on horizontal channels 
is most likely related to a very soft seafloor site or 
an \DIFdelbegin \DIFdel{unlucky deployment situation }\DIFdelend \DIFaddbegin \DIFadd{`unlucky' deployment }\DIFaddend with a wobbling station. \DIFdelbegin \DIFdel{Stations with external pack were }\DIFdelend \DIFaddbegin \DIFadd{There appears to be a
tendency for Geomar-type
stations with external packs (}\DIFaddend ob05, ob06, ob23 and ob29\DIFdelbegin \DIFdel{.
The large noise on the y-channel of ob29 is most likely not real but 
related to an enhanced sensitivity of the y-channel, 
}\DIFdelend \DIFaddbegin \DIFadd{) to have
larger noise-levels than Hamburg-type stations with integrated sensors.
}\DIFaddend although data look ok on y.
Ob05, ob06 and ob23 have a horizontal noise level a factor 
of 1000 or more larger than our high fidelity stations.
This might indicate that the external pack was at low frequencies 
more sensitive to bottom currents that the other stations, 
although an unlucky, wobbling deployment site is a second 
possible explanation.
%A final recommendation for or against an external pack is thus difficult.

%--------------------------------------------
% psd versus time
%--------------------------------------------
To quantify the noise level for the whole deployment period
the power spectral density (PSD) of seismic records has been analysed
as a function of time.
For a 3$h$ sampling interval and \DIFdelbegin \DIFdel{a }\DIFdelend 6$h$ overlapping \DIFdelbegin \DIFdel{time-window}\DIFdelend \DIFaddbegin \DIFadd{time-windows}\DIFaddend ,
traces have been mean-removed, deconvolved to velocity 
and bandpass-filtered 
with Gaussian filters centered at $f_0=0.06, 0.12, 0.24, 0.48,
0.96$, and 1.92~Hz
and with  a one-octave bandwidth.
Then, power spectral  density has been calculated and  was 
lowpass filtered and re-sampled before plotting 
in decibel (db).
% $db$, i.e.  $10log\, a^2$) 
For detailed analysis, three 
center frequencies have been selected for three Hamburg-type stations deployed in 
the 
Tyrrhenian Sea (Fig.~\ref{powerenvelope1})
and the
North Atlantic (Fig.~\ref{powerenvelope2}).
The PSD for $f_0 = 0.06\, Hz$ represents the low-noise notch
below the microseismic noise peak, 
$f_0 = 0.24\, Hz$ samples the oceanic wave generated microseismic noise, 
and 
$f_0 = 1.92\, Hz$ lies above the classical microseismic peak.
For teleseismic seismological studies often 
the frequency range 
from 0.01~Hz to 2~Hz is of interest.

The two lower traces in the low noise notch passband 
in Fig.~\ref{powerenvelope1}a (Tyrrhenian sea deployment, 
ob10 and ob11 at $0.06$ Hz)
show the expected behaviour; 
the background noise level is relatively small and clearly exceeded
by signals of teleseismic earthquakes (narrow peaks). 
The earthquake peaks can be associated with surface wave energy.
Thus,  Fig.~\ref{powerenvelope1}a gives 
the possibility to quantify and compare detection 
thresholds for surface waves. 
$M \geq 7$ events exceed the noise level about nearly 50 db, 
and 
$M \geq 5.8$ events still about 20 db.
\DIFdelbegin %DIFDELCMD < \remark{THE DIFFERENCE BETWEEN
%DIFDELCMD < M=7 AND M=6 EVENTS SHOULD BE 20 DB FROM DEFINITION OF THE MAGNITUDE
%DIFDELCMD < SCALE.  SO PROBABLY THESE NUMBERS HAVE UNCERTAINTY OF 10 DB}
%DIFDELCMD < %%%
\DIFdelend (measured in amplitude of ground velocity, the SNR is 
300 and 10, respectively). 
\\
Station ob08
was characterized by relatively large long-period noise on the 
horizontal components. 
Although the traces have been tilt-corrected by the technique described 
above, the vertical background noise is still enhanced at ob08
(upper trace in Fig.~\ref{powerenvelope1}a).
We believe that most of this low-frequency noise was current-induced.
Since ob08 was 
\DIFdelbegin \DIFdel{technical }\DIFdelend \DIFaddbegin \DIFadd{technically }\DIFaddend identical to ob10 and ob11, 
the enhanced noise at ob08 
indicates local current variations 
or an unlucky falling site of the station.

In the passband $0.24\, Hz$  
the microseismic noise is the 
dominant signal for all three stations
(Fig.~\ref{powerenvelope1}b, 
\DIFdelbegin \DIFdel{PSD }\DIFdelend \DIFaddbegin \DIFadd{the PSD is }\DIFaddend a factor of \DIFaddbegin \DIFadd{approximately }\DIFaddend 4 larger \DIFaddbegin \DIFadd{than for 0.06 Hz}\DIFaddend ). 
Single, isolated noise events are visible as broad peaks and can be associated 
with single \DIFdelbegin \DIFdel{''}\DIFdelend \DIFaddbegin \DIFadd{``}\DIFaddend storms'' or \DIFdelbegin \DIFdel{''}\DIFdelend \DIFaddbegin \DIFadd{``}\DIFaddend high swell events'' in the 
Mediterranean Sea or the Atlantic.
A noise peak may cover \DIFaddbegin \DIFadd{the }\DIFaddend range of 40 $db$ on the vertical channels.
Since the absolute noise level is increased compared to the low noise notch, 
signals from $M \geq 7$ earthquakes are detectable in \DIFdelbegin \DIFdel{few cases
only }\DIFdelend \DIFaddbegin \DIFadd{only a few cases
 }\DIFaddend (e.g. No 1, No 6).
The microseismic noise is very similar and  
correlated between the stations that were 30 and 
40 km apart.


In the 1.92 $Hz$ passband  
(Fig.~\ref{powerenvelope1}c)
the background noise is  \DIFdelbegin \DIFdel{again smaller }\DIFdelend \DIFaddbegin \DIFadd{smaller again (by a factor of 0.1
}\remark{????} \DIFadd{compared to the 0.06 Hz noise)}\DIFaddend .
Nevertheless,
signals from teleseismic earthquakes are 
often not detectable because of attenuation
along their travel path 
and because
large earthquake have 
limits on the radiation of high frequency energy.
However, regional and local earthquake signals are apparent 
in Fig.~\ref{powerenvelope1}c as correlated narrow peaks.
The noise levels follow a clear 
pattern,
and appear to be correlated between the three stations, which are 30 and 40 km apart.
The nearly continuous occurrence and the \DIFdelbegin \DIFdel{''}\DIFdelend several-days width
\DIFdelbegin \DIFdel{'' 
}\DIFdelend of noise peaks indicate that the noise
is ocean wave generated. 
However, the noise peaks at 1.92 $Hz$ do not correlate with the 
microseismic noise peaks 
at 0.24 $Hz$ and must therefore have different generation areas.
We demonstrate below that the noise at about $1\, Hz$ 
is generated by local oceanic waves above the stations, 
while the larger noise at 
0.24 $Hz$ \DIFdelbegin \DIFdel{may stem }\DIFdelend \DIFaddbegin \DIFadd{stems }\DIFaddend from regional or even teleseismic source areas.

Fig.~\ref{powerenvelope2} shows the same analysis for the North Atlantic and 
the deployment in 2002.
The three Hamburg-type stations have \DIFdelbegin \DIFdel{still }\DIFdelend an identical design and identical
sensors \DIFdelbegin \DIFdel{. 
However, the frame holding the sensor sphere of ob28 
was built from GRP and was therefore more compliant than 
the original aluminum frame of ob21 and ob26}\DIFdelend \DIFaddbegin \DIFadd{to the Tyrrhenian Sea experiment}\DIFaddend . 
In Fig.~\ref{powerenvelope2}a only 
the data of ob21 are of comparable quality to those recorded
in the Tyrrhenian \DIFdelbegin \DIFdel{sea}\DIFdelend \DIFaddbegin \DIFadd{Sea}\DIFaddend . 
Signals from $M\geq 6$ earthquakes are well detected in 
the low noise frequency band at $0.06\, Hz$. \DIFdelbegin \DIFdel{The }\DIFdelend \DIFaddbegin \DIFadd{As discussed above,
the }\DIFaddend enhanced noise level for station ob26 and ob28 
again reflects current-induced noise together 
with sensor misalignment.
\DIFdelbegin \DIFdel{Additionally at ob28, 
we assume that the compliant GPR frame 
was more sensitive to current transients 
enhancing the noise there
}\DIFdelend \DIFaddbegin \remark{Point about GRP was already mentioned on the previous page}
\DIFaddend 

The microseismic noise levels (Fig.~\ref{powerenvelope2}b)
appear again highly correlated, although the stations have been 
up to 400 \, km apart.
In contrast to the observation in the Tyrrhenian Sea the 
storm peaks are not isolated and strong microseismic 
noise seems to be continuously excited.
The absolute noise level is also higher than in the 
Tyrrhenian Sea, so that 
even the earthquake M=7.3 event is barely
visible in this frequency range.
This \DIFdelbegin \DIFdel{result is already indicated in 
}\DIFdelend \DIFaddbegin \DIFadd{observation is confirmed by
}\DIFaddend Fig~\ref{fig:tilt-correction1}c
\DIFdelbegin \DIFdel{showing }\DIFdelend \DIFaddbegin \DIFadd{which shows }\DIFaddend that the 
SNR of earthquake data has been about 1 in the North Atlantic 
for frequencies above 0.1 $Hz$.  

The absolute noise level in the North Atlantic at 
$1.92\, Hz$
is comparable to that in the Tyrrhenian Sea.
Only 
few local and regional \DIFdelbegin \DIFdel{''}\DIFdelend \DIFaddbegin \DIFadd{``}\DIFaddend earthquake peaks'' show up
in Fig.~\ref{powerenvelope2}c.
The noise levels are not correlated to the noise at $0.24\, Hz$.
However, in contrast to the Tyrrhenian Sea the 
$1.92\,Hz$ noise levels are also  not correlated between  the 
stations. 
This indicates \DIFdelbegin \DIFdel{that the 
}\DIFdelend \DIFaddbegin \DIFadd{different 
}\DIFaddend local generation areas \DIFdelbegin \DIFdel{are in this frequency range 
not correlated and have a smaller extend }\DIFdelend \DIFaddbegin \DIFadd{for each station with an extent less }\DIFaddend than the
smallest \DIFdelbegin \DIFdel{distance between the stations 
}\DIFdelend \DIFaddbegin \DIFadd{inter-station distance
}\DIFaddend ($\approx 150 \, km$). 


%--------------------------------------------
% detection thresholds
%--------------------------------------------
\DIFdelbegin \DIFdel{A question  of interest is }\DIFdelend \DIFaddbegin \DIFadd{We now address the  question  }\DIFaddend whether a brief transient signal, e.g. \DIFdelbegin \DIFdel{the }\DIFdelend \DIFaddbegin \DIFadd{a
}\DIFaddend P-wave \DIFaddbegin \DIFadd{arrival}\DIFaddend , 
can be detected in a narrow frequency band against a background of continuous 
noise.
Since the P-wave has a finite energy and hence zero power when 
averaged over all time, 
we compare the root mean square noise in a given frequency band 
($\sqrt{psd} \Delta f$, where $\Delta f$ is one-octave band) 
with model amplitudes of P-waves in that band
(Fig.~\ref{high_and_low}).
The three noise curves \DIFdelbegin \DIFdel{belong to the one }\DIFdelend \DIFaddbegin \DIFadd{represent the first }\DIFaddend quartile, median and \DIFdelbegin \DIFdel{three 
}\DIFdelend \DIFaddbegin \DIFadd{third
}\DIFaddend quartile as estimated from the whole deployment period 
(earthquakes removed) and the high fidelity stations (ob10 and ob21).
To calculate model amplitudes of body waves in teleseismic distances we 
follow the procedure described in 
the appendix of \cite{webb:98}, where 
a shallow earthquake in $70^{\circ}$ epicentral distance and 
an attenuation time of \DIFdelbegin \DIFdel{$t^{\star} = 1\, sec$ }\DIFdelend \DIFaddbegin \DIFadd{$t^{\star} = 1\, s$ }\DIFaddend has been assumed.
Higher values of
$t^{\star}$ would attenuate amplitudes at higher frequencies
\cite[see][for comparison and discussion]{webb:98}.
For tomography \DIFdelbegin \DIFdel{and other studies one would 
in principle 
}\DIFdelend \DIFaddbegin \DIFadd{one would 
}\DIFaddend like to analyse the teleseismic
body-waves at frequencies as high as possible.
However, a typical range is between 
0.02 and 2 $Hz$.
Body-wave arrivals can most likely be detected when 
their signal \DIFaddbegin \DIFadd{amplitude }\DIFaddend exceeds the noise \DIFaddbegin \DIFadd{amplitude }\DIFaddend by a factor of about 6 
(16 $db$).
\DIFdelbegin \DIFdel{Surface waves, which are not discussed, have larger 
amplitudes than body waves and thus lower detection thresholds
\mbox{%DIFAUXCMD
\cite[see also][]{webb:98}
}%DIFAUXCMD
.
}\DIFdelend 

Assuming that the several month deployments are representative for a whole 
year, detection probabilities can be 
derived from Fig.~\ref{high_and_low} and are 
discussed for distinct frequencies. 
At 0.1 $Hz$ and for the best stations, 
the body-wave \DIFdelbegin \DIFdel{of }\DIFdelend \DIFaddbegin \DIFadd{from }\DIFaddend a $M_W = 5$ event in about $70^{\circ}$ 
can be detected with a probability of about 75\% for both \DIFdelbegin \DIFdel{, }\DIFdelend the North Atlantic 
and the Tyrrhenian Sea.
Large differences occur at 0.25 $Hz$. 
In the North Atlantic, the probability to see a  
$M_W = 7.5$, 7.0 and 6.8  event is 75\%, 50\% and 25\%. \DIFaddbegin \remark{This
is not the same as what appears in the figure where the difference
between first and third quartile appears to correspond to less than
0.5 magnitude units and also seems
small when compared to figures 6,7 }
\DIFaddend In the Tyrrhenian Sea the 
$M_W = 6.0$ event would be detected in 75\% of the cases
and a 5.5 with 50\% probability. 
At 1 $Hz$ the North Atlantic is apparently preferential to the Tyrrhenian 
Sea, because high-frequency microseismic noise is relatively large 
in the Tyrrhenian Sea.
So, a 
$M_W = 6.0$ event is detected during 
about 50\% of the deployment days.
However, the prediction error of earthquake amplitudes is relatively 
large at 1 $Hz$ so that estimates are more uncertain.
The detection threshold at frequencies above 1 $Hz$ is even 
more difficult to estimate since body-wave amplitudes are more variable there 
depending on $t^{\star}$ and other factors.  \DIFaddbegin \DIFadd{Assuming that a
signal-to-noise ratio of 20 db is required for analysis of surface
wave dispersion, the threshold for surface waves to be observed 75%DIF >  of
the time is $M_W=$... at 0.02 Hz and $M_W=$... at 0.05 Hz on vertical
component measurements.  
}\remark{fill in numbers after putting surface wave lines on figure} 
\DIFaddend 

The noise level measured at sites on the East Pacific Rise 
\cite[see][figure 2]{webb:98}
is slightly higher than that observed in the North Atlantic.

\DIFdelbegin \DIFdel{Since hydrophones measure the omnidirectional pressure, the 
pressure noise is typically larger than the noise on 
vertical seismometer channels. 
}\DIFdelend In Fig.~\ref{high_and_low_hyd} the pressure noise is 
compared to body-wave model amplitudes. 
The amplitudes of the theoretical pressure signals in the water 
have been  calculated by multiplying the vertical particle
velocity with $1000 \, kg/m^3 \, \cdot \, 1500\, m/s$.
In the North Atlantic, the probability to detect a 
$M_W = 7.5$ event ($70^{\circ}$) with a hydrophone 
at frequencies above 0.2 $Hz$ is clearly less than 25\%.
The high-frequency boundary for the 25\%  probability is 
shifted to about 1 $Hz$ in the Tyrrhenian Sea.
\DIFaddbegin \remark{Surface wave thresholds for hydrophone
data probably don't make sense in our case, as the sensitivity of the hydrophones is
low in the surface wave frequency band (except for the DPG)}
\DIFaddend 

The estimates of 
Fig.~\ref{high_and_low} and
Fig.~\ref{high_and_low_hyd} 
agree well with our experience when looking at the data
and  \DIFdelbegin \DIFdel{to }\DIFdelend the general picture derived from  
\DIFdelbegin \DIFdel{figs}\DIFdelend \DIFaddbegin \DIFadd{Figs}\DIFaddend . \ref{powerenvelope1} and \ref{powerenvelope2}.  \DIFaddbegin \DIFadd{However, the
detection threshold at low frequencies can be increased by
unfavorable site conditions by up to one magnitude step (for corrected
vertical components). }\remark{KANN MAN DAS SO SAGEN?}
\DIFaddend 


%---------------------
\section{Oceanic gravity waves and noise generation}
%---------------------
Microseismic noise between $0.1$ and $1\, Hz$ is generated by 
oceanic gravity waves coupling parts of their energy in the 
seafloor and thereby exciting elastic waves.  \DIFdelbegin \DIFdel{It has been observed that most energy of microseismic noise at land stations is traveling as Rayleigh waves.
It is also well accepted that the larger peak in the 
microseismic noise spectra, the secondary peak at about 0.25 $Hz$, 
is generated by a nonlinear frequency doubling effect from 
standing oceanic gravity waves pressurizing the seafloor. 
This 
excitation mechanism is efficient in shallow as well as deep water and may 
therefore occur far from the coastlines.
However, several studies indicate that the generation areas of secondary microseismic noise in the North Atlantic is concentrated 
to few coastal regions off-shore Norway, Ireland and Scotland
\mbox{%DIFAUXCMD
\cite[e.g.][]{essen:03}
}%DIFAUXCMD
.
}\DIFdelend \DIFaddbegin \DIFadd{Microseismic noise on
land stations travels predominantly  as Rayleigh wave and can be
observed far from its generation areas \mbox{%DIFAUXCMD
\citep{essen:03}
}%DIFAUXCMD
.
}\remark{Paragraph removed because it largely repeats information from
the introduction}
\DIFaddend 

Fig.~\ref{worldaverage_20_rot} 
shows the average annual waveheight of oceanic gravity waves in 
\DIFdelbegin \DIFdel{2002.
The average waveheights have been }\DIFdelend \DIFaddbegin \DIFadd{2002 }\DIFaddend calculated from 
global oceanic wave models \cite[WAM, e.g.][]{komen:94} \DIFdelbegin \DIFdel{sampling }\DIFdelend \DIFaddbegin \DIFadd{which sample }\DIFaddend the sea waveheight in 
six hour intervals and $1^{\circ}$ grid points.
The largest oceanic waves, and thus the largest microseismic noise, can be expected
south of $45^{\circ}$ latitude.
Noise conditions should be optimal in the equatorial regions.
In the North Atlantic and North Pacific  the noise level is expected
to be significantly lower during the summer, since 
the average height of oceanic waves is lower during the summer.
The figure shows that the expected noise is much lower at the
Galapagos hotspot, the Seychelles or Cape Verdes compared to
Iceland or off-shore South Chile.
The Mediterranean Sea and especially the Agean Sea should have one
of the lowest microseismic noise levels.

The observed noise and the estimated 
detection
thresholds 
in the North Atlantic and the Tyrrhenian Sea
correlate well with the average oceanic waveheights
in Fig.~\ref{worldaverage_20_rot}. 
South of Iceland in the North Atlantic the average annual waveheight 
is about three meters. It is only half a meter or less in the 
Tyrrhenian Sea.
Thus, a comparative approach can be used to 
roughly predict 
expected detection thresholds in other regions 
of the \DIFdelbegin \DIFdel{worlds }\DIFdelend \DIFaddbegin \DIFadd{world's }\DIFaddend oceans;
for example, latitudes below \DIFdelbegin \DIFdel{the }\DIFdelend $42^{\circ}$S have the largest
average waveheights of four  meters and more, and are thus expected
to have the worst microseismic noise.

\DIFaddbegin \DIFadd{The WAM model waveheights  of the North Atlantic were calculated by
the Deutsche Wetterdienst in Offenbach, Germany, and have been 
sampled at three hour intervals and 
$0.75^{\circ}$ grid points. 
}\DIFaddend In order to locate the generation areas of microseismic noise in the 
North Atlantic we correlate the \DIFaddbegin \DIFadd{WAM }\DIFaddend oceanic waveheight 
with the square-root of
PSD time series for different passbands. 
as plotted in 
\DIFdelbegin \DIFdel{figs}\DIFdelend \DIFaddbegin \DIFadd{Figs}\DIFaddend .  \ref{powerenvelope1} and \ref{powerenvelope2}.
\DIFdelbegin \DIFdel{The WAM model waveheights  of the North Atlantic have been 
sampled at three hour intervals and 
$0.75^{\circ}$ grid points. 
They were calculated by the Deutsche Wetterdienst in Offenbach, Germany.
Fig.~\ref{wheightcomp} 
}\DIFdelend For example, the 1 $Hz$ vertical channel 
noise at ob21 and the waveheight directly above this station 
correlates well with the sea waveheights above the station
(Fig.~\ref{wheightcomp}), 
whereas the noise at $0.24\, Hz$ does not, even though it is $\sim$100 times stronger.
We assume that a good correlation of oceanic waveheights and noise 
psd, more precisely its square root, 
over the recording period of three and a half months
indicates a major noise generation area.
This assumption is justified when the major noise generation areas for seismic noise at a given station remain the same throughout
the experiment.
Thus, 
Fig.~\ref{wheightcomp} indicates 
that the secondary microseismic noise at ob21 
is not generated locally.

To estimate the major generation areas of noise at ob21 we
calculated and plotted linear correlation coefficients between the 
measured noise on the vertical channel and 
sea waveheights (Fig.~\ref{wavecorr1}).
% A coefficient of zero and one indicates no and perfect correlation, respectively.
This correlation technique has been applied successfully 
on land stations 
to estimate generation areas of secondary microseismic noise
in North Europe
\cite[][]{essen:03}.
We find that the secondary microseismic noise at ob21 
has been mainly generated off-coast of Ireland an Scotland
(Fig.~\ref{wavecorr1}a), while the 
noise at 1 $Hz$ was indeed generated by oceanic waves above the station.
This observation is of interest since  it predicts that a large portion
of the noise signals at 0.24 $Hz$ are likely to 
arrive as plane waves at the station 
and may therefore be attenuated by applying array methods.

Applying the same analysis to the other deployed stations 
showed partly similar patterns but also revealed further
generation areas.
For instance, ob24 has the highest 
correlation coefficient to a source region on the Mid-Atlantic Ridge 
(MAR) 
off-coast South Iceland, while oceanic waves off the coast of Ireland 
seem to have a minor influence.
To estimate all source regions of secondary microseismic noise 
relevant for the complete OBS-array we calculated an average 
noise PSD time series before correlating with oceanic 
waveheights.
Fig.~\ref{wavecorr2}a indicates that altogether three major
noise generation areas are effective, 
one off-coast West Ireland, the other off-coast South Iceland on the MAR,
and the third in a band NE of Iceland.
The correlation coefficient is largest for the region 
off-coast Ireland indicating that this  
is the strongest generation area among the three.
\remark{diese Abbildung zu gmeinsamen Generation Areas 
muss ich nochmal rechnen da Freuenzbereich
nicht genau vergleichbar -- allerdings hab ich im Moment Problme da
programm nicht mehr so laufen will wie voher}

Knowing the major generation areas of noise before a deployment 
can help to optimize the array and network design. 
Fig.~\ref{wavecorr2}b shows that the major generation areas
can be roughly estimated when averaging PSD noise-curves from 
inland stations and correlation with WAM waveheights.
In our case station BORG on Iceland and 
ESK on Scotland have been used for averaging.
More stations may be added, although our tests showed that adding stations 
will not significantly change the 
pattern in Fig.\ref{wavecorr2}b.

%---------------------
\section{Discussion and conclusion}
%---------------------
Longterm deployments in the Tyrrhenian Sea and in the North Atlantic 
have been analysed to characterize the 
seafloor broadband noise and the differences between two types
of free-fall ocean-bottom stations.
Free-fall broadband OBS have been proposed to be 
used for several large-scale 
passive seismological experiments.
Therefore, our results and analysis is of interest to 
better plan these future experiments and deployments, to 
improve the station design, the array configuration and 
the deployment technique.

%-------------------------------------
\subsubsection*{The role of the station design and current-induced noise}
%-------------------------------------

\DIFdelbegin \DIFdel{Similar }\DIFdelend \DIFaddbegin \DIFadd{Similarly }\DIFaddend to observations with other broadband \DIFdelbegin \DIFdel{obs-es 
}\DIFdelend \DIFaddbegin \DIFadd{OBS
}\DIFaddend our free-fall stations experience lowest noise 
in the low noise notch below 0.1 $Hz$.
The low noise notch is bounded by 
the microseismic noise at high frequencies\DIFdelbegin \DIFdel{and }\DIFdelend \DIFaddbegin \DIFadd{,
and by
noise from infra-gravity waves }\DIFaddend at low frequencies, 
depending on the water depth\DIFdelbegin \DIFdel{,
noise from infra-gravity waves}\DIFdelend . 
The low noise notch is of interest for 
surface wave and teleseismic long-period body wave studies.
It is likely that the 
noise level in the low noise notch is 
determined by current-induced tilt.
The ocean seismic network pilot experiment (OSNP) and MOISE experiment
 confirms this
view \cite[][]{collins:01,stephen:03,stutzmann:01}.
Three broadband seismographs have been deployed 
close to the ODP Hole 843B at 4400 m depth
located about 225 km southwest of Oahu, Hawaii.
One was deployed in a borehole 240 m beneath the seafloor,
a second was buried just below and a third on the seafloor.
Below 0.1 $Hz$ the sensor buried under the seafloor 
was 10's of $dB$ less than the the one sitting on the seafloor.
The noise level of the station sitting on the seafloor
was correlated with current speed and 
increased with current speed 
by about 8 $dB / cm/s$ on the horizontal component.

For our high fidelity  stations the 
average noise level in the low-noise notch at 0.08 $Hz$ was about
$10^{-15}$ and 
$10^{-13}\, m^2/s^4 / Hz$ on vertical and horizontal 
seismometer channels, respectively,
and thus comparable or slightly worse than 
''quiet day'' measurements of the broadband seafloor station of the OSNP 
experiment, which showed 
values at this frequency of about
$10^{-16}$ and 
$10^{-13.5}\, m^2/s^4 / Hz$.
\cite[see][]{stephen:03}.
The current-induced noise depends strongly on the deployment site, e.g. whether the 
station is deployed on a steep slope or in a sea-bottom bathymetric valley. 
However,  \DIFdelbegin \DIFdel{we think that the large }\DIFdelend \DIFaddbegin \DIFadd{the differences in noise level by a }\DIFaddend factor up to
10000\DIFaddbegin \remark{on page 7 (old manuscript) the factor is 1000. I guess
the 10000 refers to GEOMAR stations where we are unsure about the gain
so it might be better to say ``factor of 1000 and more'' here} \DIFaddend in noise level at the different 
stations may also indicate \DIFdelbegin \DIFdel{feasible and less feasible current sensitivities of stations
or wobbling stations}\DIFdelend \DIFaddbegin \DIFadd{differences in station design that
determine susceptibility to currents or promote `wobbling'}\DIFaddend .
For instance, the deployment of an external pack is recommended  by many 
seismologists to reduce noise from a possible vibration of the station frame.
However, from the long-term deployments with different stations types we can conclude that 
stations with external packs have never achieved a lower horizontal low-frequency 
noise level than those with \DIFaddbegin \DIFadd{a }\DIFaddend frame-integrated sensor. \DIFdelbegin \DIFdel{The }\DIFdelend \DIFaddbegin \DIFadd{At the same
time, the }\DIFaddend frame-integrated stations are \DIFdelbegin \DIFdel{, however, more easy }\DIFdelend \DIFaddbegin \DIFadd{easier }\DIFaddend to handle and less sensitive to mechanical
malfunction.
Another noise problem \DIFaddbegin \DIFadd{we }\DIFaddend identified is the wobbling of single stations for unlucky
\DIFdelbegin \DIFdel{when possibly fallen to difficult deployment sites }\DIFdelend \DIFaddbegin \DIFadd{station sites due to the uncontrolled nature of free-fall deployments}\DIFaddend .
This suggests that a \DIFdelbegin \DIFdel{video-controlled launching of obs-ses }\DIFdelend \DIFaddbegin \DIFadd{more controlled launching with video support }\DIFaddend has the potential to further 
reduce the current-induced noise.

The current-induced tilt noise is 
mainly felt by horizontal sensors.
However, if the sensor is badly levelled the horizontal 
tilt-noise is transferred to the vertical channel.
We have estimated misalignments of several degrees (\DIFaddbegin \DIFadd{possibly }\DIFaddend $34^{\circ}$ in 
one case \DIFdelbegin \DIFdel{?}\DIFdelend \DIFaddbegin \remark{Do we really believe this?}\DIFaddend ).
In conclusion, a further improved leveling mechanics 
\DIFdelbegin \DIFdel{of our obs-ses 
}\DIFdelend will significantly reduce cross-coupling and thus low-frequency noise on \DIFaddbegin \DIFadd{the
}\DIFaddend vertical channels.

We further were able to demonstrate that tilt-induced noise \DIFaddbegin \DIFadd{on the
vertical  }\DIFaddend can be 
efficiently \DIFdelbegin \DIFdel{attenuated }\DIFdelend \DIFaddbegin \DIFadd{reduced }\DIFaddend when horizontal seismometer data are available.
The increase in SNR for amplitudes was about a factor of 200 or more 
at noisy stations, showing the importance to measure \DIFdelbegin \DIFdel{three channels}\DIFdelend \DIFaddbegin \DIFadd{with
three-component seismometers}\DIFaddend .

%-------------------------------------
\subsubsection*{Character and generation of microseismic seafloor noise}
%-------------------------------------

Narrowband microseisms with a peak frequency at about 
$0.24\, Hz$ \DIFdelbegin \DIFdel{is }\DIFdelend \DIFaddbegin \DIFadd{are }\DIFaddend the dominant  noise \DIFaddbegin \DIFadd{recorded on our free-fall obs.
They are  strongly dependent }\DIFaddend on \DIFdelbegin \DIFdel{our OBS-ses.
It is strongly depend on }\DIFdelend the experiment region.
\DIFdelbegin \DIFdel{E.g.}\DIFdelend \DIFaddbegin \DIFadd{For example}\DIFaddend , in 
the North Atlantic the microseismic noise-power
is about a factor of at least 100 larger than in the Tyrrhenian Sea.
The large amplitudes and the frequencies of the 
microseismic noise can be explained by a nonlinear fluid-dynamic 
effect in the oceanic layer, when 
standing oceanic gravity waves pressurize the seafloor    
at \DIFdelbegin \DIFdel{the double }\DIFdelend \DIFaddbegin \DIFadd{twice the }\DIFaddend frequency of the oceanic waves, which have 
\DIFdelbegin \DIFdel{larger }\DIFdelend \DIFaddbegin \DIFadd{their largest }\DIFaddend amplitudes at about $0.12\, Hz$
(the spectrum of water waves depends on different factors and waves are
dispersive).
By analyzing land stations, 
\cite{essen:03} have shown that 
the majority of secondary microseismic noise in the North Atlantic 
is generated at \DIFaddbegin \DIFadd{a }\DIFaddend few sites near  \DIFdelbegin \DIFdel{to }\DIFdelend the coast of Norway, Scotland.
This result is further confirmed and extended  by our study. 
We identify additional generation areas of microseismic noise 
for the North Atlantic 
off-shore the \DIFdelbegin \DIFdel{North-West coast }\DIFdelend \DIFaddbegin \DIFadd{north-west coast of }\DIFaddend Ireland/Scotland, 
on the Reykjanes Ridge south of Iceland,
and in a band between North-Iceland and West-Norway.
The generation areas are dominant presumably because 
the submarine topography and the preferential wind-wave directions
are particularly favourable towards conversion of oceanic wave energy 
into secondary microseisms.

\DIFdelbegin \DIFdel{We further were able to give estimates of these generation areas by means 
of seismological land data and oceanic wave models. This is an important result because it allows to estimate the 
 dominant noise strength and }\DIFdelend \DIFaddbegin \DIFadd{Using the same technique, similar generation areas were found for the
microseismic noise on land stations near the experimental region (BORG
and ESK). This means
both the 
 strength and dominant }\DIFaddend wave direction of \DIFdelbegin \DIFdel{noise }\DIFdelend \DIFaddbegin \DIFadd{microseismic noise can be
 estimated  }\DIFaddend prior to an \DIFaddbegin \DIFadd{ocean bottom }\DIFaddend experiment.
Better estimates \DIFdelbegin \DIFdel{of }\DIFdelend \DIFaddbegin \DIFadd{for required }\DIFaddend deployment times can be \DIFaddbegin \DIFadd{thus }\DIFaddend derived for temporary deployments and 
the station configuration might be optimized to \DIFdelbegin \DIFdel{attenuate }\DIFdelend \DIFaddbegin \DIFadd{suppress }\DIFaddend microseismic noise
by means of frequency-wavenumber filtering.

A second finding is that microseismic noise at about $1\, Hz$, 
which is smaller than the noise at 0.25 $Hz$,
is generated by same nonlinear oceanic wave interaction 
but locally above the station \DIFaddbegin \DIFadd{by wind-driven waves}\DIFaddend .
This has been postulated before \DIFaddbegin \DIFadd{\mbox{%DIFAUXCMD
\citep[e.g.][]{babcock:94}
}%DIFAUXCMD
}\DIFaddend , but is evidenced here with 
a new technique incorporating oceanic waveheights.
\DIFdelbegin \DIFdel{E.g., 
\mbox{%DIFAUXCMD
\cite{babcock:94}
}%DIFAUXCMD
analyzed environmental data from ocean surface buoys and an ocean bottom 
array deployed off-coast North Carolina in the North Atlantic. 
They studied the relationship between meteorological surface conditions 
and seismic noise on the seafloor. 
They found that the microseismic single- ($\approx 0.01\, Hz$) and 
double-frequency  peak ($0.16 - 0.3\, Hz$) 
was highly variable during the time of the experiment, and is 
probably generated at coastlines at teleseismic distances. 
The small-energy high-frequency end of the microseismic noise was, however, 
apparently generated by the interaction of local high-frequency oceanic waves.
\mbox{%DIFAUXCMD
\cite{webb:98}
}%DIFAUXCMD
reviewed the climatology of high-frequency microseismic noise
and gave examples of good correlation of high-frequency microseismic noise 
with local wind speed above the station.
Wind-driven oceanic waveheights are included in the WAM models we used
to derive frequency-dependent generation areas.
}\DIFdelend \DIFaddbegin \remark{I removed the last paragraph because I think it contains too
  much detail for conclusion section}
\DIFaddend 


%--------------------------------------
\subsubsection*{Detection thresholds and 
necessary deployment times}
%--------------------------------------
Due to the long deployment times we were able to 
give estimates of the detection probabilities for 
P-waves of teleseismic earthquakes.
As a reference we choose events with different magnitudes \DIFdelbegin \DIFdel{in }\DIFdelend \DIFaddbegin \DIFadd{at
}\DIFaddend $70^{\circ}$ \DIFdelbegin \DIFdel{and 
two stations having }\DIFdelend \DIFaddbegin \DIFadd{epicentral distance and the 
two stations with }\DIFaddend the best fidelity.
Below the microseismic noise \DIFdelbegin \DIFdel{noise }\DIFdelend the detection thresholds 
for both deployments \DIFdelbegin \DIFdel{was }\DIFdelend \DIFaddbegin \DIFadd{were }\DIFaddend similar; 
75\% probability to detect a $M_W = 5.5$ event.

Large differences were found in the microseismic frequency band.
While we predict a detection probability of 
75\% for an $M_W =7.5$ event in $70^{\circ}$ distance in the North 
Atlantic, a   
$M_W =6.0$ event is measured in the Tyrrhenian Sea with the same 
probability.
On the hydrophone, the 
$M_W =7.5$ event can be recorded in less than 25\% of the time.
The very low detection threshold on the hydrophone channel 
in the microseismic band is in full agreement with our experience. 
It demonstrates the importance \DIFdelbegin \DIFdel{to deploy full obs-se }\DIFdelend \DIFaddbegin \DIFadd{of deploying full obs }\DIFaddend with four channels 
instead of \DIFdelbegin \DIFdel{hydrophones alone 
}\DIFdelend \DIFaddbegin \DIFadd{only hydrophones 
}\DIFaddend in the North Atlantic or at other noisy places.


To summarize our experience with free-fall \DIFdelbegin \DIFdel{obs-ses we think 
that improved and optimized stations that will be deployed by 
a controlled launching system }\DIFdelend \DIFaddbegin \DIFadd{obs, 
free-fall stations }\DIFaddend are usable for broadband seismological 
purposes \DIFdelbegin \DIFdel{. }\DIFdelend \DIFaddbegin \DIFadd{although a more consistent quality could be achieved by
a controlled launching system and improved station design . }\remark{Note that I changed the content of the previous
  sentence somewhat - please check!} \DIFaddend A burial of sensors would most likely further 
reduce 
the noise level in the low noise notch, \DIFaddbegin \DIFadd{particularly for the
horizontal components, }\DIFaddend but will not help in the 
microseismic frequency band \DIFdelbegin \DIFdel{were largest noise occurs}\DIFdelend \DIFaddbegin \DIFadd{where the largest noise levels occur}\DIFaddend .
However, since noise generation areas are far-distant in many cases 
small-scale arrays to attenuate microseismic noise 
would probably help.


%----------------------------------------------------------
%   Acknowledgements
%\begin{acknowledgments}
%\end{acknowledgments}
\subsection*{Acknowledgements}
The work was supported by DFG, the Leitstelle f\"ur Mittelgrosse
Forschungsschiffe and the University of Hamburg.  We thank the crews
of the FS Alexander-von-Humboldt and the Bjarni Saemundson for their
professional work under difficult sea conditions.  Roger Sitko, J\"org
Reinhardt, Martin Hensch, Andreas Wittwer, J\"org Petersen,
and Viola Gaw helped with the cruises.  Ernst
Fl\"uh and J\"org Bialas provided invaluable advice on technical
aspects of the instrumentation, Rudi Widmer-Schniedrig provided the
global WAM models and the DWD (Deutscher Wetterdienst) the WAMS in the North Atlantic.


%----------------------------------------------------------
%   Appendix
%----------------------------------------------------------
\section*{Appendix A}
In order to calculate cross-coupling transfer functions, 
we first selected a number of noise sequences which are free of
larger events and recording artifacts such as spikes or clipped
waveforms, in total between 18 and about 60~h worth of data. We then cut each
noise sequence into windows of e.g. 164~s length with 50\% overlap between
adjacent windows.  Each window is mean removed and 
tapered with a Bartlett window to reduce spectral leakage \citep{press:92}.
The transfer function between `source' component $s$ and `response'
component $r$, $T_{rs}(\omega)$, is then calculated in the frequency domain:
\begin{displaymath}
T_{rs}(\omega)=\frac{<A_s^*(\omega)A_r(\omega)>}{<A_s^*(\omega)A_s(\omega)>}
\end{displaymath}
where $A_s(\omega)$ and $A_r(\omega)$ are the complex spectra of the
source and response components, respectively, and $<\ldots>$
implies averaging over all windows of all sequences (in the following
equations the dependence on $\omega$ is assumed but not explicitly stated).
The amplitude of the transfer function thus obtained is set to zero at
those frequencies where coherency drops below 0.7\DIFdelbegin \DIFdel{and the }\DIFdelend \DIFaddbegin \DIFadd{. The }\DIFaddend coherency is defined as
\begin{displaymath}
\gamma_{rs}=\frac{|<A_r^*A_s>|}{\sqrt{<A_r^*A_r>}\sqrt{<A_s^*A_s>}}\ \ 
{\bf .}
\end{displaymath}
In a small transition region near the threshold the amplitudes are reduced.  
The predicted signal for component $r$ is then $T_{rs} A_s$, 
\DIFdelbegin \DIFdel{i
}\DIFdelend which is subtracted from the observed signal $A_r$ to obtain the corrected 
\DIFdelbegin \DIFdel{i
}\DIFdelend signal $A_{r'}$.  
Following \citet{crawford:00}, we carry out the following steps to correct 
the vertical component $z$ using the horizontal components $x$ and $y$.
\begin{eqnarray*}
1. & A_{z'}= & A_z-T_{zx}A_x \\
   & A_{y'}= & A_y-T_{yx}A_x \\
2. & A_{z''}= & A_{z'}-T\DIFdelbegin \DIFdel{_{z'x'}}\DIFdelend \DIFaddbegin \DIFadd{_{z'y'}}\DIFaddend A_{y'}
\end{eqnarray*}
An equivalent technique can be applied to correct noise correlated
between the vertical and pressure component, $p$.
\begin{eqnarray*}
3. & A_{z'''}= & A_{z''}-T_{z''P}A_p
\end{eqnarray*}
where we assume that the pressure noise is not correlated with the
horizontal components. 

Figures~\ref{fig:tilt-correction1}
and~\ref{fig:tilt-correction2} illustrate the application of the
algorithm to ob28, the station where it was most effective. The
coherence between horizontal  and vertical components is high
below $\sim 0.12$Hz (Fig.~\ref{fig:tilt-correction1}b), and coherence
between the pressure and vertical components is high in the
microseismic band (0.15--0.3~Hz). The transfer function between the
vertical and the $y$ component exhibits a relatively flat amplitude
response and a nearly constant phase below 0.12~Hz (Fig.~\ref{fig:tilt-correction1}c), which implies that
the noise on the vertical component has the same shape as that of the
horizontal  components but smaller amplitudes (factor 0.11 for \DIFdelbegin \DIFdel{$z/x$
}\DIFdelend \DIFaddbegin \DIFadd{$z/y$
}\DIFaddend and -0.05 for \DIFdelbegin \DIFdel{$z/y$}\DIFdelend \DIFaddbegin \DIFadd{$z/x$}\DIFaddend , clearly visible in
the time domain, too (Fig.~\ref{fig:tilt-correction2}a,b).
\DIFdelbegin %DIFDELCMD < \remark{SHOULD CHANGE THIS TO Y COMPONENT TO BE ABLE TO DISCUSS LARGE
%DIFDELCMD <   AMPLITUDE VALUES} 
%DIFDELCMD < %%%
\DIFdelend If interpreted purely as tilt and there is no
difference between the gain for different channels, the amplitude of
the transfer function is the tangent of the tilt angle, in this case
implying a tilt angle of $7^{\circ}$. \DIFdelbegin %DIFDELCMD < 

%DIFDELCMD < \remark{besser nicht so viel begruenden, es kann gut sein dass 
%DIFDELCMD < ein Teil des Tilts wegen fehlenden Ausgleichsgewichten 
%DIFDELCMD < und Kontrolle ! am Sensor entstanden sind}
%DIFDELCMD < %%%
\DIFdelend \DIFaddbegin \DIFadd{Tilts for the other stations are
listed in Table~\ref{tilt_table}.
}\DIFaddend Another potential source of coherent noise
could be oscillations of the frame.  As mentioned above, it is hard to
determine the cause of the coherent signal without additional
information.  The transfer function above
0.12~Hz is unreliable because the coherency is low, and it is not used
for further processing.

After subtracting the predicted, tilt-induced signal from 
the vertical, 
the noise PSD is reduced by a factor of 100-1000
in the coherent band 
(Fig.~\ref{fig:tilt-correction1}d) 
but remains above the noise level for the
quietest instrument, 
 which was ob21 during the deployment in the 
North Atlantic.  
The microseismic noise psd is only
marginally reduced by the pressure-based correction.

 
%----------------------------------------------------------
%   Bibliography
%----------------------------------------------------------
%\bibliographystyle{../../style/tectonophysics}
%\bibliography{../../bib/bib}
\begin{thebibliography}{}

\bibitem[Babcock et~al., 1994]{babcock:94}
Babcock, J., Kirkendall, B., and Orcutt, J., 1994.
\newblock Relationship between ocean bottom noise and the environment.
\newblock Bull. Seism. Soc. Am., 84:1991--2007.

\bibitem[Collins et~al., 2001]{collins:01}
Collins, J., Vernon, F., Orcutt, J., Stephen, R., Peal, K., Woodings, F.,
  Spiess, F., and Hildebrand, J., 2001.
\newblock Broadband seismology in the oceans: lessons from the {Ocean Seismic
  Network Pilot Experiment}.
\newblock JGR, 28:49--52.

\bibitem[Crawford and Webb, 2000]{crawford:00}
Crawford, W. and Webb, S., 2000.
\newblock Identifying and removing tilt noise from low-frequency ($< 0.1\, hz$)
  seafloor vertical seismic data.
\newblock Bull. Seism. Soc. Am., 90:952--963.

\bibitem[Dahm et~al., 2002]{dahm:02}
Dahm, T., Thorwart, M., Flueh, E., Braun, T., Herber, R., Favali, P.,
  Beranzoli, L., D'Anna, G., Frugoni, F., and Smirglio, G., 2002.
\newblock Ocean bottom seismological instruments deployed in the {Tyrrhenian
  Sea}.
\newblock EOS Transaction, 83:309, 314.

\bibitem[Duennebier and Sutton, 1995]{duennebier:95}
Duennebier, F.~K. and Sutton, G., 1995.
\newblock Fidelity of ocean bottom seismic observations.
\newblock Marine Geophysical Researches, 17:535--555.

\bibitem[Essen et~al., 2003]{essen:03}
Essen, H.-H., Kr\"uger, F., Dahm, T., and Grevemeyer, I., 2003.
\newblock On the generation of secondary microseisms observed in northern and
  central {Europe}.
\newblock J. Geophys. Res., 108:10.1029/2002JB0002338.

\bibitem[Flueh and Bialas, 1999]{flueh:99}
Flueh, E. and Bialas, J., 1999.
\newblock Ocean bottom seismometers.
\newblock Sea Technology, 40:41--46.

\bibitem[Godin and Chapman, 1999]{godin:99}
Godin, O. and Chapman, D., 1999.
\newblock Shear-speed gradients and ocean seismo-acoustic noise resonances.
\newblock J. Acoust. Soc. Am., 106:2367--2382.

\bibitem[Hasselmann, 1963]{hasselmann:63}
Hasselmann, K., 1963.
\newblock A statistical analysis of the generation of microseisms.
\newblock Rev. Geophys., 1:177--210.

\bibitem[Hedlin and Orcutt, 1989]{hedlin:89}
Hedlin, M. and Orcutt, J., 1989.
\newblock A comparative study of island, seafloor, and subseafloor ambient
  noise levels.
\newblock Bull. Seism. Soc. Am., 79:172--179.

\bibitem[Komen et~al., 1994]{komen:94}
Komen, G., Cavaleri, L., Donelan, M., Hasselmann, K., Hasselmann, S., and
  Janssen, P., 1994.
\newblock Dynamics and modelling of \DIFdelbegin \DIFdel{ocan }\DIFdelend \DIFaddbegin \DIFadd{ocean }\DIFaddend waves.
\newblock Cambridge Univ. Press, New York, pp. 1--532.

\bibitem[Longuet-Higgins, 1950]{longuet-higgins:50}
Longuet-Higgins, M., 1950.
\newblock A theory of the origin of microseisms.
\newblock Phil. Trans. Royal Soc. Lon. A, 243:2--36.

\bibitem[Peterson, 1993]{peterson:93}
Peterson, J., 1993.
\newblock Observation and modelling of seismic background noise.
\newblock USGS Open File Report, 93-322:1--49.

\bibitem[Press et~al., 1992]{press:92}
Press, W., Flannery, B., Teukolsky, S., and Vetterling, W., 1992.
\newblock Numerical recipes, 2nd. edition.
\newblock Cambridge University Press.

\bibitem[Ritsema and Allen, 2003]{ritsema:03}
Ritsema, J. and Allen, R.~M., 2003.
\newblock The elusive mantle plume.
\newblock Earth Planet. Sci. Lett., 207(1--4):1--12.

\bibitem[Schreiner and Dorman, 1990]{schreiner:90}
Schreiner, A. and Dorman, L., 1990.
\newblock Coherency lengths of seafloor noise: effects of ocean bottom
  structure.
\newblock J. Acoust. Soc. Am., 88:1503--1514.

\bibitem[Stephen et~al., 2003]{stephen:03}
Stephen, R., Spiess, F., Collins, J., Hildebrand, J., Orcutt, J., Peal, K.,
  Vernon, F., and Wooding, F., 2003.
\newblock Ocean seismic network of pilot experiment.
\newblock Geochemistry, Geophysics, Geosystems, 4:doi:10.1029/2002GC000485.

\bibitem[Stutzmann et~al., 2001]{stutzmann:01}
Stutzmann, E., Montagner, J.-P., Sebai, A., Crawford, W., Thirot, J.-L.,
  Tarits, P., Stakes, D., Romanowicz, B., Karczewski, J.-F., Koenig, J.-C.,
  Savary, J., Neuhauser, D., and Etchemedy, S., 2001.
\newblock Moise: A prototype multiparameter ocean-bottom station.
\newblock Bull. Seism. Soc. Am., 91:885--891.

\bibitem[Webb, 1998]{webb:98}
Webb, S., 1998.
\newblock Broadband seismology and noise under the ocean.
\newblock Rev. Geophys., 36:105--142.

\bibitem[Webb and Crawford, 1999]{webb:99}
Webb, S. and Crawford, W., 1999.
\newblock Seafloor seismology and deformation under ocean waves.
\newblock Bull. Seism. Soc. Am., 89:1535--1542.

\end{thebibliography}
%--------------------------------------------------------------------------

%----------------------------------------------------------
%   Figures
%----------------------------------------------------------

%----------------------------------------------------------
\begin{figure}
\centerline{\includegraphics[width=14cm]{island2.ps}}
\caption{OBS stations deployed in the North Atlantic between 
April and July 2002.
The open circles indicate stations which had technical problems 
with sensors and power consumption and 
therefore operated for only a few days.
The station 
marked by the black circle
was not recovered for unknown technical reasons.
Stations 20, 21, 26 and 28 were of Hamburg-type, 
the others of GEOMAR-type. 
The isolines declare water depth in meter.
Volcanic centres and fissure zone in Iceland are indicated by lines.
See text for further explanations. 
}
\DIFaddbeginFL \remark{should really have (b) Tyrrhenian Sea deployment map.  But
  Figure can be small with (a) and (b) next to each other}
\DIFaddendFL \label{station_map}
\end{figure}
%----------------------------------------------------------

%----------------------------------------------------------
\begin{figure}
\centerline{\includegraphics[angle=-90.0,width=14cm]{fig2.ps}}
\caption{Recordings 
of the Kamchatka, 28 June 2002, M=7.3 event 
at the OBS deployed in the 
North Atlantic. 
BORG is the vertical channel of the permanent broadband station in Reykjavik, Iceland,
and is plotted for comparison.
The traces show
available vertical and hydrophone channels
 indicated by $z$
and $h$, respectively.
Hamburg-type stations are indicated by HH, GEOMAR-type stations by GM.
A bandpass filter with corner frequencies at 
$0.025$ and $0.1 \, Hz$ has been 
applied. 
Additional, tilt-induced long-period noise has been removed for 
ob26, ob28 and ob23.
All traces are normalized to their own maxima.
Station ob22 \DIFdelbeginFL \DIFdelFL{shows spurious signals of unknown origin, }\DIFdelendFL \DIFaddbeginFL \DIFaddFL{recorded the P wave }\DIFaddendFL but \DIFaddbeginFL \DIFaddFL{the latter part of the wave train
}\DIFaddendFL is \DIFdelbeginFL \DIFdelFL{included for 
completeness}\DIFdelendFL \DIFaddbeginFL \DIFaddFL{spoilt by spikes (resulting in spurious signal in the bandpass
filtered record)}\DIFaddendFL .
}
\label{kamchatka_M7.3}
\end{figure}
%----------------------------------------------------------

%----------------------------------------------------------
\begin{figure}
\centerline{\includegraphics[angle=-90.0,width=14cm]{kodiak_10_6.7b1.ps}}
\caption{Recordings 
of the Kodiak Island, 10 January 2001, M=6.7 shallow event 
at the OBS deployed in the 
Tyrrhenian Sea. 
The epicentral distance was $84^{\circ}$.
The traces show three component seismograms recorded on a 
GEOMAR-type station (ob05) and on two Hamburg-type stations 
(ob10, ob11). 
Horizontal seismograms have been rotated to 
radial and
transverse directions,
and 
traces are normalized to their own maxima.
A lowpass filter with a corner frequency of
0.05\, Hz has been applied.
}
\label{quake_tysea}
\end{figure}
%----------------------------------------------------------

%----------------------------------------------------------
%\begin{figure}
%\centerline{\includegraphics[angle=-90.0,width=14cm]{tilt_improvement_ob28.ps}}
%\caption{
%Comparison of raw and tilt-corrected data on the vertical 
%channel of ob28 and ob26 for the 
%Kamchatka 2002 event. 
%Both horizontal components are plotted for reference.
%The numbers declare the maximum value of the traces in thousand counts.
%}
%\label{tilt_improvement}
%\end{figure}
%----------------------------------------------------------

%----------------------------------------------------------
\begin{figure}
\includegraphics[angle=-90,width=15cm]{ftilmann/wav-2002-06-22-02-ob28.ps}
\includegraphics[width=9cm]{ftilmann/snr-ob28.ps}
\caption{(a) Seismogram example for a magnitude 6.5 event in $50^{\circ}$
epi-central distance. The top trace shows the \DIFdelbeginFL \DIFdelFL{uncorrected }\DIFdelendFL \DIFaddbeginFL \DIFaddFL{corrected }\DIFaddendFL vertical \DIFaddbeginFL \DIFaddFL{component}\DIFaddendFL ,
the \DIFdelbeginFL \DIFdelFL{middle traces }\DIFdelendFL \DIFaddbeginFL \DIFaddFL{second from top trace shows }\DIFaddendFL the \DIFdelbeginFL \DIFdelFL{two horizontal components}\DIFdelendFL \DIFaddbeginFL \DIFaddFL{tilt-corrected vertical component}\DIFaddendFL ,
and the bottom \DIFdelbeginFL \DIFdelFL{trace
}\DIFdelendFL \DIFaddbeginFL \DIFaddFL{traces }\DIFaddendFL the \DIFdelbeginFL \DIFdelFL{tilt-corrected vertical component}\DIFdelendFL \DIFaddbeginFL \DIFaddFL{two horizontal components}\DIFaddendFL . All traces are normalised to
their own maximum.  Note that the uncorrected vertical and horizontal
waveforms are very similar, but the amplitude of the latter is much
higher. Letters next to the waveform indicate predicted arrival times
of some seismic phases. (b)  Magnification of the P wave
arrival for this event.  (c) Signal-to-noise ratios (SNR) for corrected and
uncorrected vertical channels estimated from  five
events 
(\DIFaddbeginFL \DIFaddFL{at }\DIFaddendFL distances 
\DIFaddbeginFL \DIFaddFL{of }\DIFaddendFL $14^{\circ}$,$49^{\circ}$,$60^{\circ}$,$72^{\circ}$, and $112^{\circ}$\DIFdelbeginFL \DIFdelFL{g}\DIFdelendFL ). 
The SNR was
estimated by calculating and comparing PSDs for a window
immediately preceding the event, and a window encompassing the event
(beginning with the P wave arrival and encompassing most of the
surface wave train). The dashed line (tilt- and pressure-corected) and
the dotted line lie on top of each other for most frequencies. The SNR
for station ob21 is included for reference (gray line).}
\label{fig:tilt-correction2}
\end{figure}
%----------------------------------------------------------

%----------------------------------------------------------
\begin{figure}
\centerline{
%\includegraphics[width=7.3cm]{fig6a.ps}
\includegraphics[width=7.3cm]{psd_comparison1-isla.ps}
\hskip1cm
%\includegraphics[width=7.3cm]{fig6b.ps}
\includegraphics[width=7.3cm]{psd_comparison1-tysea.ps}
}
\caption{
Average \DIFaddbeginFL \DIFaddFL{vertical }\DIFaddendFL power spectral density \DIFdelbeginFL \DIFdelFL{at three }\DIFdelendFL \DIFaddbeginFL \DIFaddFL{of  }\DIFaddendFL Hamburg-type stations 
deployed in the North Atlantic (a) and 
Tyrrhenian Sea (b), 
where a time window of 85 (julian day from 104 to 189)
and 146 days (day 337 in 2000 to 117 in 2001) has been used.
Teleseismic earthquake signals ($M > 6$)  have been removed 
before averaging.
Local earthquakes have not been removed and may \DIFdelbeginFL \DIFdelFL{have }\DIFdelendFL \DIFaddbeginFL \DIFaddFL{make }\DIFaddendFL a small
contribution to noise above 2 Hz.
Sensors ob29 and ob06 had an unknown amplification factor and their 
average psd curves 
were therefore scaled down to fit the microseismic peaks of 
nearby calibrated stations.
The range between the USGS low and high noise model 
(Petersson 1993) is indicated by gray shading.
The deployment depth is indicated in meter.
\DIFaddbeginFL \remark{You should state whether this are corrected or uncorrected
  data - they look uncorrected but I wasn't sure}
\DIFaddendFL }
\label{psd_comparison}
\end{figure}
%----------------------------------------------------------

%----------------------------------------------------------
\begin{figure}
%\centerline{\includegraphics[width=14cm]{spectrowiggle_thy.ps}}
\centerline{\includegraphics[width=14cm]{fig4.ps}}
\caption{
Ground velocity power 
in 6 h overlapping time windows
at three Hamburg-type stations 
deployed in the 
Tyrrhenian Sea
(vertical seismometer channel, power spectral density plots in $db$, 
$10\log\, \dot{u}^2$). 
a) gives {\em psd} for the `noise notch' frequency band around $f_0 = 0.06\, Hz$, 
b) for the microseismic frequency band centered at $f_0 = 0.24\, Hz$, 
and c) \DIFdelbeginFL \DIFdelFL{chg}%DIFDELCMD < {%%%
\DIFdelendFL for the high frequency microseismic noise band
\DIFdelbeginFL %DIFDELCMD < }
%DIFDELCMD < %%%
\DIFdelendFL at $f_0 = 1.9\, Hz$. 
The scale spans 50 $db$ for every trace. The power is 
 \DIFdelbeginFL \DIFdelFL{a }\DIFdelendFL 20 db (a factor of 100)
larger for the microseismic frequency band in b).\DIFaddbeginFL \remark{In the figure
  itself a factor of x4 is written, so it is in the text.  Is this an
  inconsistency. If not, I don't understand where the factor 100 is
  coming from}
\DIFaddendFL The deployment depth was 1550 $m$, 1893 $m$ and 2569 $m$ for 
ob08, ob11, and ob10, respectively.
Occurrence times of teleseismic earthquakes with $M >5.8$ are indicated by 
marker lines
in a), where numbers indicate earthquakes with $M \geq 7$
}
\label{powerenvelope1}
\end{figure}
%----------------------------------------------------------

%----------------------------------------------------------
\begin{figure}
%\centerline{\includegraphics[width=14cm]{spectrowiggle_0.06_0.24.ps}}
\centerline{\includegraphics[width=14cm]{fig5.ps}}
\caption{
Ground velocity power at three Hamburg-type stations 
deployed in the 
North Atlantic.
The station depths for stations ob21, ob26, and
ob28 are 2780~m, 1389~m, and 2268~m, respectively. 
The squared oceanic wave height at station ob21 is indicated as dotted 
line in c).
See Fig.~\ref{powerenvelope1} for further explanations.
}
\label{powerenvelope2}
\end{figure}
%----------------------------------------------------------

%----------------------------------------------------------
\begin{figure}
\centerline{
%\includegraphics[width=8cm]{fig7a.ps}
%\includegraphics[width=8cm]{fig7b.ps}
\includegraphics[width=8cm]{psd_high_low_vera.ps}
\includegraphics[width=8cm]{psd_high_low_verb.ps}
}
\caption{
Models of  the \DIFdelbeginFL \DIFdelFL{amplitudes in }\DIFdelendFL vertical 
acceleration  \DIFaddbeginFL \DIFaddFL{amplitudes }\DIFaddendFL in one-octave bands\DIFaddbeginFL \remark{Figure label says
  there are 1/3 octave bands!} \DIFaddendFL of \DIFaddbeginFL \DIFaddFL{surface waves and of }\DIFaddendFL the P-wave 
from earthquakes at a distance in $70^{\circ}$
and an assumed frequency independent 
attenuation time of $t^{\star} = 1 \, sec$
\cite[see][for further description]{webb:98}.
Also shown are the \DIFdelbeginFL \DIFdelFL{one }\DIFdelendFL \DIFaddbeginFL \DIFaddFL{first }\DIFaddendFL quartile, median and \DIFdelbeginFL \DIFdelFL{three }\DIFdelendFL \DIFaddbeginFL \DIFaddFL{third }\DIFaddendFL quartile 
of vertical acceleration noise 
($\sqrt{psd} \Delta f$) \DIFaddbeginFL \DIFaddFL{at the highest fidelity stations
}\DIFaddendFL in the 
North Atlantic (a, ob21) and 
Tyrrhenian Sea (b, ob10).
\DIFdelbeginFL \DIFdelFL{The stations with the highest fidelity are shown 
(ob21 and ob10) and times of }\DIFdelendFL \DIFaddbeginFL \DIFaddFL{Windows containing large }\DIFaddendFL earthquakes have been removed from 
data streams.\DIFaddbeginFL \
\DIFaddendFL }
\label{high_and_low}
\DIFaddbeginFL \remark{To add surface
waves to the figure, after Webb,98, Fig. 7: draw straight lines in
log-log plot:
%\begin{flushleft}
Units (Hz, m/s$^2$) 
M$_W$=6 (1e-3,2e-10) - (0.7,1e-5);
M$_W$=7 (1e-3,1.1e-8) - (0.25,1e-5);
M$_W$=8 (1e-3,3.5e-7) - (0.07,1e-5);
%\end{flushleft}
(Numbers manually read off plot so not very accurate)
Cut off lines at 0.05 Hz}

\DIFaddendFL \end{figure}
%----------------------------------------------------------

%----------------------------------------------------------
\begin{figure}
\centerline{
\includegraphics[width=8cm]{psd_high_low_hyda.ps}
\includegraphics[width=8cm]{psd_high_low_hydb.ps}
}
\caption{
Models of the amplitudes in pressure 
in one-octave bands of the teleseismic 
P-waves  ($70^{\circ}$)
are compared to one quartile, median and 
three quartile of 
pressure noise in the 
North Atlantic (a, ob21) and Tyrrhenian Sea (b, ob10)
See Fig.~\ref{high_and_low} for further description.
}
\label{high_and_low_hyd}
\DIFaddbeginFL \remark{This figure has a problem with the period scale: 1 s period
  seems to correspond to 2 Hz!}
\DIFaddendFL \end{figure}
%----------------------------------------------------------

%----------------------------------------------------------
\begin{figure}
\centerline{\includegraphics[angle=-90.,width=16cm]{worldaverage_20.ps}}
\caption{Average oceanic waveheight in 2002.
The frequency of occurence of wave heights is plotted for selected locations
as histograms,
where the upper histograms corresponds to the period from April to September,
and the lower one from October to March, respectively.
}
\label{worldaverage_20_rot}
\end{figure}
%----------------------------------------------------------


%----------------------------------------------------------
\begin{figure}
\centerline{
\includegraphics[width=16cm]{wheightcomp_ob21.ps}
}
\caption{
}
\label{wheightcomp}
Comparison of seismic seafloor noise (continuous line, square root of noise psd) and 
oceanic waveheight (dashed line) at the position of ob21.
A Gauss filter center-frequency of 0.24 and 0.96 $Hz$ has been 
applied to \DIFaddbeginFL \DIFaddFL{the }\DIFaddendFL seismic data in (a) and (b), respectively.
\end{figure}
%----------------------------------------------------------

%----------------------------------------------------------
\begin{figure}
\centerline{
\includegraphics[width=9.0cm]{correlation_ob21_0.24Hz.ps}
\hskip-1cm
\includegraphics[width=9.0cm]{correlation_ob21_0.96Hz.ps}
}
\caption{
Linear correlation coefficient between the 
oceanic waveheight field
and the 
power of 
filtered vertical ground-velocity at station ob21 
(black filled circle)
bandpass filtered
at $f_0 = 0.24\, Hz$ (a) 
and
$0.96\, Hz$ (b).
}
\label{wavecorr1}
\end{figure}
%----------------------------------------------------------

%----------------------------------------------------------
\begin{figure}
\centerline{
\includegraphics[width=9.0cm]{correlation_obs_sum.ps}
\hskip-1cm
\includegraphics[width=9.0cm]{correlation_land_sum.ps}
}
\caption{
Superposed 
correlation coefficients from all OBS stations deployed (a)
and from the two land stations BORG and ESK for comparison
(b).
The station locations are indicated by black filled circles.
The bandpass filter had a center frequency at the 
secondary microseismic noise peak 
($f_0 = 0.24\, Hz$).
}
\label{wavecorr2}
\end{figure}
%----------------------------------------------------------

%----------------------------------------------------------
\begin{figure}
\includegraphics[width=7.2cm]{ftilmann/psd-ob28.ps}\hfill\includegraphics[width=7.2cm]{ftilmann/coh-ob28.ps}
\includegraphics[width=7.2cm]{ftilmann/trans-ob28.ps}\hfill\includegraphics[width=7.2cm]{ftilmann/psdcor-ob28.ps}
\caption{Tilt correction example for station OB28 (a) Noise PSD
for all components. \DIFdelbeginFL \DIFdelFL{The scale is arbitrary, but horizontal and vertical components are
normalised by the same constant. }\DIFdelendFL Note that the horizontal channel has
the same shape as the vertical but is noisier by  10-20 db. (b)
Coherency between the vertical noise and the pressure and two horizontal
channels. (c) The empirically determined transfer function from the
horizontal $x$ component
to the vertical component.  
Convolved with the noise on the $x$ component, this function predicts the noise on the vertical
channel.
%Dotted lines give formal 95\% confidence intervals resulting
%from the transfer function estimation.
(d) Noise PSD for the
vertical component.  For the tilt-corrected signal (dashed line),
the predicted noise from the horizontal components has been subtracted
from the vertical components. For the tilt and pressure-corrected
signal (dotted line), the predicted noise from the pressure signal has
been subtracted
from the tilt-corrected signal.  For comparison, the uncorrected noise
PSD for OB21, the \DIFdelbeginFL \DIFdelFL{least noisy }\DIFdelendFL \DIFaddbeginFL \DIFaddFL{quietest }\DIFaddendFL station, has been superimposed (gray
line). Below $\sim$0.12 Hz, the dashed and dotted line lie on top of
each other, above this frequency the dotted and continuous line lie on
top of each other.
\DIFdelbeginFL %DIFDELCMD < \remark{PLOTS (a,d)WOULD BE NEATER IF INSTRUMENT CORRECTION WERE
%DIFDELCMD < CARRIED OUT AND DECIBEL SCALE USED.  PLEASE COMMENT WHETHER YOU THINK
%DIFDELCMD < THIS IS NECESSARY. 
%DIFDELCMD < Yes, I think instrument correction is better. We are already showing 
%DIFDELCMD < psd fo corrected data, and I refer to this figure as an 
%DIFDELCMD < example of psd on horizontal channels: it is important to see here 
%DIFDELCMD < how flat the psd ae below 0.1 Hz}
%DIFDELCMD < %%%
\DIFdelendFL }
\DIFaddbeginFL \remark{The figure is still the old one, not the latest one I said
  (with true acceleration spectra rather than counts).  I have changed
  the figure caption to describe the new figure.}
\DIFaddendFL \label{fig:tilt-correction1}
\end{figure}
%----------------------------------------------------------

\end{document}

