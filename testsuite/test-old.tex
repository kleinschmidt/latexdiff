\documentclass[11pt]{article}
%\usepackage[germanb]{babel}
\usepackage{epsf}
\input{mylatex.sty}
\input{myepsf.sty}
\setlength{\topmargin}{-0.2in}
\setlength{\textheight}{9.5in}
\setlength{\textwidth}{6.5in}
\setlength{\oddsidemargin}{0.0in}
%\setlength{\topmargin}{1.0in}
\renewcommand{\baselinestretch}{1.8}

\newcommand{\ea}{{\it et al.}}
\newcommand{\pd}[2]{\frac{\partial #1}{\partial #2}}

\newenvironment{reflist}{\begin{list}{}{\setlength{\itemindent}{-\leftmargin}} \small }
                           {\end{list}}

\providecommand{\remark}[1]{{ \bf [ \em #1 ]}}


%\def\rf{\vskip 5pt\par\hang\noindent}
\def\rf{\item}

\newcommand{\jgr}{J. Geophys. Res.}
\newcommand{\refer}[6]{{#1, #2, #3, {\it #4} {\bf #5}, {#6}.}}
% \refer{authors}{year}{title}{Journal}{Volume}{page}
\newcommand{\refartbook}[7]{{#1, #2, #3. In: {\it #4}, ed. by #5, #6, #7.}}
% \refartbook{authors}{year}{article title}{book-title}{editor}{publisher}{pages}
\newcommand{\refbook}[5]{{#1, #2, {\it #3}, #4.}}
% \refbook{authors}{year}{book-title}{publisher}{page}
% Note: currently output of page is suppressed
\newcommand{\refreport}[5]{{#1, #2, {\it #3}, #4, #5.}}
% \refreport{authors}{year}{report-title}{Report Number}{Publisher}


\title{DRAFT - A different kind of discrimination: a trigger algorithm for detecting natural seismicity in the presence of artificial sources}
\author{Frederik J. Tilmann}
\date{\today}

\begin{document}
\maketitle

\begin{abstract}
ABSTRACT
\end{abstract}

\section[Introduction]{Introduction}

\mbox{An old mbox}
The use of automatic triggers in order to detect earthquake activity
is routine in both temporary and permanent networks.   After suitable pre-processing of
the data, e.g. frequency filtering for noise suppresion, the
triggering algorithm generally proceeds in two stages
(Figure~\ref{fig:flowchart}).  In the first stage, each trace is considered
individually, and the trigger algorithms marks the times, at which
signals or arrivals are recorded, usually by comparing the short-term
average with the long-term average of the signal (STA/LTA method), and
triggering when this ratio exceeds a certain threshold.  This
algorithm results in many spurious triggers due to noise bursts or
cultural or electronic noise, unless the station is very quiet.  For
this reason, a second stage compares the trigger times of all stations
of the network and only declares a valid trigger if more than a
specified number of stations have declared a trigger at almost the
same time (network triggering), excluding thus noise bursts at single stations.  
\begin{verbatim}
some ticky text
\end{verbatim}
However, this type of algorithm is unable to distinguish artificial
sources that are strong enough to be recorded at all stations from
natural seismicity. For example, during deployments of networks of ocean bottom
seismometers and hydrophones, often active shooting is carried out
contemporaneously with surveys of microearthquake activity, such that
the continuous recordings of earthquake activity are `contaminated' with artificial signals.
% either
%directly above the network in order to better locate the network
%stations and characterise crustal structure beneath them, or simply
%nearby, because collaborating groups carry out reflection
%or wide-angle surveys.   
Unfortunately, the frequency content of
airgun shots and many other marine sources is not sufficiently removed from the typical frequency
content of local earthquakes such that suppression of the shots by
pre-filtering is not possible.    Nevertheless, shots are immediately
apparent on plots of continuous data (Figure~\ref{fig:data})
simply because of their regularity; marine shooting is nearly always
carried out at regular intervals, with shot intervals between
$\sim$10 and 120~s.   This fact led to the idea to develop
an algorithm that can automatically discriminate shots from
earthquakes based on the regularity of triggers.  This algorithm will
be described in the following.  It is operating between station-wise STA/LTA
triggering and network triggering, detecting and removing shots from
the station-wise trigger lists.   Earthquakes occuring between shots
will in most cases be triggering correctly.

A sentence with *** stars*in*my*eyes.

  As the shot times are known it would, of course, also be possible to
  exclude the times of active shooting 
from the trigger algorithm, or even to selectively block the
triggers that coincide with expected arrival times of the shots at the
various stations.  However, this would be tedious and would require
good communication between the different groups, in particular if the
instruments are deployed and recovered on different cruises.
Excluding times of shooting completely would also result in valid
earthquakes that occurred between shots to be excluded
unneccessarily.  Blocking triggers selectively based on {\it a priori}
knowledge of the exact shooting times and locations would require theoretical travel times
from each shot point to all stations of the network and puts even
higher demands on the quality of communication.


\section{Algorithm}

\section{Data examples}

\section{Discussion}

\section*{References}

\begin{figure}
\caption{Flow chart (a) Network triggering (b) Shot removal stage}
\label{fig:flowchart}
\end{figure}

\begin{figure}
\caption{Trace including shots and earthquakes (a) Data high-pass filtered to remove microseismic noise (corner 5.0 Hz) (b) Trigger trace causes triggers at both events and shots \protect\remark{from GERSHWIN data}}
\label{fig:data}
\end{figure}

\begin{figure}
\caption{Histogram of distances to neighbouring trigger times.  The
  peak near zero comes about because often the coda of an event
  triggers repeatedly (because of different phases or multiples). The
  peaks at multiples of .. s are due to regular shot intervals.}
\label{fig:histograms}
\end{figure}


\end{document}
